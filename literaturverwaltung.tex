% !TeX root = lfgw.tex
\chapter{Literatur und Zitate automatisch verwalten}
\dictum[U. Eco, Name der Rose]{...}
\label{biblatex}
\autor{Lukas C. Bossert}

\section{Der (neue) Standard: biblatex und biber}

\paket{biblatex}
Vgl.\footcite[79\psqq]{rouquette:2012}
\footcite{voss:bibliografien}
\footcite{wassenhoven:dtk2008/2}
\footcite{wassenhoven:dtk2008/4}



\section{Aufbau der Bibliografie-Datenbank}

\subsection{Grundlegender Aufbau}

\subsection{Schlüsselvergabe}

\subsection{Publikationstypen und ihre Datenfelder}

\subsubsection{Bücher (Monografie, Sammmelband)}

\subsubsection{Zeitschriften (Artikel, Rezensionen)}

\subsubsection{Lexika, Handbücher}

\subsubsection{Internet-Ressourcen}

\subsubsection{Übersicht: Publikationstypen und Datenfelder}

alle auflisten....


\section{Zitate}

Normale Zitate werden mit dem Befehl \cs{cite} ausgeführt:
\begin{lfgwcode}{label={lis:XXX}}
\cite*@\oarg{Präfix}\oarg{Suffix}\marg{Schlüssel}%@*
\end{lfgwcode}

Während \meta{Präfix}  eine kurze Anmerkung \emph{vor} die Ztation (z.\,B. \enquote{Vgl.}) setzt, 
wird  \meta{Suffix} für gewöhnlich für Seitenzahlen verwendet.
Ist nur ein optionales Argument definiert, 
dann wird es als \oarg{Suffix} behandelt.
\begin{lfgwcode}{label={lis:XXX}}
\cite*@\oarg{Suffix}\marg{Schlüssel}%@*
\end{lfgwcode}
Der \meta{Schlüssel} korrespondiert mit dem Schlüssel der Bibliografie-Datei.

\begin{lfgwexample}{label={lis:xxx}}
\enquote{Beware of bugs in the above code; I have only proved it correct, not tried it.} Donald Knuth
Der öffentliche Raum ist Teil einer Stadt \cite{Osland2016}.
\end{lfgwexample}

\minisec{\cs{cites}}
Möchte man hingegen mehrere Autoren oder Werke zitieren, 
gibt es zwei Möglichkeiten:
Entweder kann man dies durch die Kommagetrennte Reihung der \marg{Schlüssel} machen,
was jedoch den Nachteil hat, dass man für die einzelnen \marg{Schlüssel} keine \oarg{Präfixe} oder \oarg{Suffixe} definieren kann.

Die andere Möglichkeit sieht vor, den Befehl \cs{cites} zu verwenden, 
bei dem für jeden Autor/ jedes Werk sowohl \oarg{Präfixe} als \oarg{Suffixe} definiert werden kann.
Zudem lässt sich für die gesamte Reihung ein \oarg{Präfix} und ein \oarg{Suffix} festlegen:
\begin{lfgwcode}{label={lis:XXX}}
\cites(Prä-Präfix)(Suf-Suffix)
  *@\oarg{Präfix}\oarg{Suffix}\marg{Schlüssel}@*%
  *@\oarg{Präfix}\oarg{Suffix}\marg{Schlüssel}@*%
  *@\oarg{Präfix}\oarg{Suffix}\marg{Schlüssel}\ldots@*
\end{lfgwcode}
\begin{lfgwexample}{label={lis:xxx}}
Der öffentliche Raum ist Teil einer Stadt \cites(vgl.)(){Osland2016}{Evangelidis2014}.
\end{lfgwexample}
 
\minisec{\cs{parencite}}
Manchmal soll die Zitation in Klammern stehen.
Um dies nicht händisch in runde Klammern setzen zu müssen (und ggf. die ›Klammerschachtelregel‹ zu verletzen),
kann dafür der Befehl  \cs{parencite} verwendet werden:
\begin{lfgwcode}{label={lis:XXX}}
\parencite*@\oarg{Suffix}\marg{Schlüssel}%@*
\end{lfgwcode} 
Mit diesem Zitationsbefehl wird die korrekte Ordnung von korrespondierenden Klammern berücksichtigt.
\begin{lfgwexample}{label={lis:xxx}}
\enquote{Der öffentliche Raum ist Teil einer Stadt.} \parencite{Osland2016}
\end{lfgwexample}

\minisec{\cs{parencites}}
Ebenso lassen sich auch mehrere Zitationen mit Klammern umschließen.
Dies wird mittels \cs{parencites} umgesetzt:
\begin{lfgwcode}{label={lis:XXX}}
\parencites(Prä-Präfix)(Suf-Suffix)%
*@\oarg{Präfix}\oarg{Suffix}\marg{Schlüssel}@*%
*@\oarg{Präfix}\oarg{Suffix}\marg{Schlüssel}@*%
*@\oarg{Präfix}\oarg{Suffix}\marg{Schlüssel}\ldots@*
\end{lfgwcode}
\begin{lfgwexample}{label={lis:xxx}}
Der öffentliche Raum ist Teil einer Stadt.\parencites(s.)(){Osland2016}%
[vgl.][]{Evangelidis2014}.
\end{lfgwexample}

\minisec{\cs{textcite}}
Neben den bereits angeführen \cs{cite}- Befehlen gibt es eine dritte Möglichkeit der Zitationsangabe:
\cs{textcite} ist vor allem für die Fälle zu nutzen, 
bei denen der Autor/ das Werk im Fließtext genannt sein soll, 
aber die weiteren Angaben (Publikationsjahr, Seitenangabe) nur in runden Klammern dahinter.
\begin{lfgwcode}{label={lis:XXX}}
\textcite*@\oarg{Suffix}\marg{Schlüssel}%@*
\end{lfgwcode} 

\begin{lfgwexample}{label={lis:xxx}}
Der öffentliche Raum ist Teil einer Stadt, sagt \textcite{Osland2016}.
\end{lfgwexample}

\minisec{\cs{textcites}}
Wiederum können mehrere Autoren/ Werke mittels  \cs{textcites} gelistet werden:
\begin{lfgwcode}{label={lis:XXX}}
\textcites(Prä-Präfix)(Suf-Suffix)%
  *@\oarg{Präfix}\oarg{Suffix}\marg{Schlüssel}@*%
  *@\oarg{Präfix}\oarg{Suffix}\marg{Schlüssel}@*%
  *@\oarg{Präfix}\oarg{Suffix}\marg{Schlüssel}\ldots@*
\end{lfgwcode}
\begin{lfgwexample}{label={lis:xxx}}
Der öffentliche Raum ist Teil einer Stadt, sagen \textcites{Osland2016}%
[vgl.][]{Evangelidis2014}.
\end{lfgwexample}


\minisec{\cs{footcite}}
Darüberhinaus gibt es weitere \cs{cite}-Befehle, 
die die Einbettung der Zitation beeinflussen. 
Zunächst kann man mit \cs{footcite} die Zitation direkt als eigene Fußnote setzen:
 \begin{lfgwcode}{label={lis:XXX}}
\footcite*@\oarg{Präfix}\oarg{Suffix}\marg{Schlüssel}@*
\end{lfgwcode}
\begin{lfgwexample}{label={lis:xxx}}
\enquote{Der öffentliche Raum ist Teil einer Stadt.}\footcite{Osland2016}
\end{lfgwexample}
\cs{footcite} ist das Äquivalent zu \lstinline/\footnote{\cite{Osland2016}.}/
was jedoch manche (überflüssige) Tipparbeit spart.

\minisec{\cs{footcites} }
Für mehrere Autoren/ Werke in einer Fußnote gibt es auch \cs{footcites}:
\begin{lfgwexample}{label={lis:xxx}}
\enquote{Der öffentliche Raum ist Teil einer Stadt.}\footcites(s.)(){Osland2016}%
[vgl.][]{Evangelidis2014}
\end{lfgwexample}
 
 
\minisec{\cs{smartcite}}
Eine klevere Art und Weise Zitationen als Fußnote zu setzen,
bitete der Befehl \cs{smartcite}.
 \cs{smartcite} reagiert auf die Umgebung des Befehls:
 Befindet sich \cs{smartcite} innerhalb des Fließtexts wird es wie \cs{footcite} behandelt 
 und die Zitation in eine Fußnote setzen. 
Wird die Zitation allerdings in einer Fußnote aufgerufen,
erfolgt die Ausgabe nach dem Schema von \cs{cite}. 
 is a clever 
\begin{lfgwcode}{label={lis:XXX}}
\smartcite*@\oarg{Suffix}\marg{Schlüssel}%@*
\end{lfgwcode} 

\begin{lfgwexample}{label={lis:xxx}}
Der öffentliche Raum ist Teil einer Stadt.\smartcite{Osland2016} 
Eventuell aber zugangsbeschränkt.\footnote{\smartcite[vgl.][]{Evangelidis2014}.}
\end{lfgwexample}


\minisec{\cs{smartcites}}
Wiederumg gibt es analog auch den Befehl \cs{textcites}, 
um mehrere Autoren/ Werke klever zu zitieren:
\begin{lfgwcode}{label={lis:XXX}}
\smartcites(Prä-Präfix)(Suf-Suffix)%
  *@\oarg{Präfix}\oarg{Suffix}\marg{Schlüssel}@*%
  *@\oarg{Präfix}\oarg{Suffix}\marg{Schlüssel}@*%
  *@\oarg{Präfix}\oarg{Suffix}\marg{Schlüssel}\ldots@*
\end{lfgwcode}
\begin{lfgwexample}{label={lis:xxx}}
Der öffentliche Raum ist Teil einer Stadt.\smartcites{Osland2016}{Evangelidis2014} 
Eventuell aber zugangsbeschränkt.\footnote{\smartcites{Osland2016}%
[cf.][]{Evangelidis2014}.}
\end{lfgwexample}

\minisec{\cs{autocite}}
Mit  \cs{autocite} ist eine individuelle und flexible Zitationsangabe möglich,
indem man in der Präambel steuert,
wie \cs{autocite} ausgegben werden soll.
Für gewöhnlich stehen folgende Optionen zur Verfügung:
\begin{labeling}{XXXXX}
	\item[plain] Ausgabe wie \cs{cite}
	\item[inline]Ausgabe wie \cs{parencite}
	\item[footnote]Ausgabe wie \cs{footcite}
\end{labeling}
\begin{lfgwcode}{label={lis:XXX}}
\autocite*@\oarg{Präfix}\oarg{Suffix}\marg{Schlüssel}%@*
\end{lfgwcode} 

\begin{lfgwexample}{label={lis:xxx}}
Der öffentliche Raum ist Teil einer Stadt \autocite{Osland2016} 
\end{lfgwexample}

\minisec{\cs{fullcite} \cs{footfullcite}}
Mit den Befehlen \cs{fullcite} und \cs{footfullcite} werden zwei Möglichkeiten gegeben,
mit denen man den kompletten Bibliographie-Eintrag in den Fließtext bzw. in die Fußnote schreiben kann.
\begin{lfgwcode}{label={lis:XXX}}
\fullcite*@\oarg{Präfix}\oarg{Suffix}\marg{Schlüssel}@*
\footfullcite*@\oarg{Präfix}\oarg{Suffix}\marg{Schlüssel}@*
\end{lfgwcode} 

\begin{lfgwexample}{label={lis:xxx}}
Der öffentliche Raum ist Teil einer Stadt.\footfullcite{Osland2016}
Das steht auch geschrieben bei \fullcite{Evangelidis2014}
\end{lfgwexample}



\minisec{\cs{citeauthor} \cs{citetitle}}
Neben den ›geläufigen‹  \cs{cite}-Befehlen kann man auch nur den oder die Autoren zitieren 
und ebenso nur den Werktitel.
Dies funktioniert für den Fließtext und für Fußnoten gleichermaßen:
\begin{lfgwcode}{label={lis:XXX}}
\citeauthor*@\oarg{Präfix}\oarg{Suffix}\marg{Schlüssel}%@*
\end{lfgwcode} 
  and for the works 
\begin{lfgwcode}{label={lis:XXX}}
\citetitle*@\oarg{Präfix}\oarg{Suffix}\marg{Schlüssel}%@*
\end{lfgwcode} 

\begin{lfgwexample}{label={lis:xxx}}
Der öffentliche Raum ist Teil einer Stadt sagt \citeauthor{Osland2016} in \citetitle{Osland2016}.
\footnote{Der öffentliche Raum ist Teil einer Stadt sagt \citeauthor{Osland2016} in \citetitle{Osland2016}.}
\end{lfgwexample}


\minisec{\cs{nocite}}
Möchte man einen Eintrag nicht als Zitation im Text haben, 
aber auf die Auflistung in der Bibliographie nicht verzichten,
dann kann man \cs{nocite}\marg{Schlüssel} verwenden.
Die Zitation wird damit ›unsichtbar‹ zitieren, sodass sie dennoch in der Bibliographie auftaucht.




\section{Bibliographiestile}

\minisec{Standardstile von biblatex}
biblatex (--> verschiedene biblatex-Stile mit deren Optionen erläutern, die für Geisteswissenschaftler relevant sind)
-- biblatex-dw (sehr viele Optionen)\\
-- archaeologie (sehr viele Optionen)\\
-- geschichtsfrkl (viele Optionen)\\
-- historische-zeitschrift\\
-- apa\\
-- historian\\
-- biblatex-fiwi\\

\minisec{Nützliche Bibliografiestile, die nachinstalliert werden müssen}

Installation als Paket nötig:
...



\section{Ein Beispiel für Historiker: Quellen und Sekundärliteratur}
\index{Quellenverzeichnis}
\index{Bibliographie}
\index{Literaturverzeichnis}
\index{Sekundärliteratur}


%\begin{lfgwcode}{}  % läuft bei mir nicht - thm
%\printbibheading[%
%  heading=bibliography,
%  %heading=bibnumbered,% if you want it numbered
%  title={Bibliographie}]% Überschrift für Bibliographie  
%\end{lfgwcode}


\begin{lfgwcode}{}
\printbibliography[%
  heading=subbibliography,
  %heading=subbibnumbered,% if you want it numbered
  keyword=ancient,%
  title={Antike Quellen}]
\end{lfgwcode}

\begin{lfgwcode}{}
\printbibliography[%
  heading=subbibliography,
  keyword=corpus,%
  title={Abkürzungen und Sigel}]

\printbibliography[%
  heading=subbibliography,
  notkeyword=ancient,%
  notkeyword=corpus,%
  title={Sekundärliteratur}]
\end{lfgwcode}

