% !TeX root = lfgw.tex
\chapter{Literatur und Zitate automatisch verwalten}
\dictum[U. Eco, Name der Rose]{...}
\label{biblatex}


\section{Der (neue) Standard: biblatex und biber}

\paket{biblatex}
Vgl.\footcite[79\psqq]{rouquette:2012}
\footcite{voss:bibliografien}
\footcite{wassenhoven:dtk2008/2}
\footcite{wassenhoven:dtk2008/4}



\section{Aufbau der Bibliografie-Datenbank}

\subsection{Grundlegender Aufbau}

\subsection{Schlüsselvergabe}

\subsection{Publikationstypen und ihre Datenfelder}

\subsubsection{Bücher}

\subsubsection{Zeitschriften}

\subsubsection{Artikel}

\subsubsection{Internet-Resourcen}

\subsubsection{Übersicht: Publikationstypen und Datenfelder}

alle auflisten....


\section{Zitate}

\lstinline/\cite{key}/

\lstinline/\footcite{key}/

\lstinline/\nocite{key}/




\section{Bibliographiestile}

\minisec{Standardstile von biblatex}


\minisec{Nützliche Bibliografiestile, die nachinstalliert werden müssen}

Installation als Paket nötig:
...



\section{Ein Beispiel für Historiker: Quellen und Sekundärliteratur}
\index{Quellenverzeichnis}
\index{Bibliographie}
\index{Literaturverzeichnis}
\index{Sekundärliteratur}
