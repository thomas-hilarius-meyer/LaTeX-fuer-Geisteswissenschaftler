% !TeX root = lfgw.tex
\chapter{Kritische Apparate setzen}
\dictum[?]{Kritik ist überall, zumal in Deutschland, nötig.}

\index{Apparat} \index{Varianten}
\label{reledmac}

\enquote{Edieren ist eine Erziehung zur Bescheidenheit [...] Es ist ferner eine Erziehung zur 
Genauigkeit, wie alle Philologie.}%
\footnote{Erich Trunz, Ein Tag aus Goethes Leben. Acht Studien zu Leben und Werk, München 1999, S.~213.}

Bezug zu TUSTEP?
Recherche zu anderen Alternativen?

\paket{reledmac}, \paket{ledpar} und \paket{ednotes} (wird nicht erklärt.)
\footcite[165\psqq]{rouquette:2012}


\chapter{Literatur und Zitate automatisch verwalten}
\dictum[U. Eco, Name der Rose]{...}
\label{biblatex}


\section{Der (neue) Standard: biblatex und biber}

\paket{biblatex}
Vgl.\footcite[79\psqq]{rouquette:2012}
\footcite{voss:bibliografien}
\footcite{wassenhoven:dtk2008/2}
\footcite{wassenhoven:dtk2008/4}



\section{Aufbau der Bibliografie-Datenbank}

\subsection{Grundlegender Aufbau}

\subsection{Schlüsselvergabe}

\subsection{Publikationstypen und ihre Datenfelder}

\subsubsection{Bücher}

\subsubsection{Zeitschriften}

\subsubsection{Artikel}

\subsubsection{Internet-Resourcen}

\subsubsection{Übersicht: Publikationstypen und Datenfelder}

alle auflisten....


\section{Zitate}

\lstinline/\cite{key}/

\lstinline/\footcite{key}/

\lstinline/\nocite{key}/




\section{Bibliographiestile}

\minisec{Standardstile von biblatex}


\minisec{Nützliche Bibliografiestile, die nachinstalliert werden müssen}

Installation als Paket nötig:
...



\section{Ein Beispiel für Historiker: Quellen und Sekundärliteratur}
\index{Quellenverzeichnis}
\index{Bibliographie}
\index{Literaturverzeichnis}
\index{Sekundärliteratur}


\chapter{Texte durch Register erschließen}
\dictum[Schiller, Räuber]{Dein Register hat ein Loch.}
\index{Register} \index{Index}

\section{Allgemeines}

Sortierproblem bei Umlauten


\section{Mehrere Register zu einem Dokument}

\minisec{Problem und grundsätzliche Strategie}
\cite{voss:einfuehrung}

\minisec{Lösungsansatz I: imakeidx}
\paket{imakeidx} 

\minisec{Lösungsansatz II: splitidx}
\paket{splitidx}

\section{Bibelstellenregister}

\label{Bibelstellenregister}
\index{Bibelstellenregister}
\paket{bibleref-german}

\section{Worthäufigkeit}
\index{Häufigkeit (von Wörtern)}

\section{Reimwörter}
\index{Reimwörterbuch}

