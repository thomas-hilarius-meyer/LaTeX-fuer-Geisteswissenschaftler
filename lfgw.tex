\documentclass[draft=false,11pt,DIV=8]{scrbook}  	% dieser DIV-Wert wird später überschrieben...
\usepackage[paperwidth=17cm,paperheight=24cm]{geometry}
\KOMAoptions{DIV=13,BCOR=7mm}


%\documentclass[12pt,BCOR=10mm]{scrbook}
\usepackage[utf8]{inputenc}
\usepackage[OT1,T1]{fontenc}
\usepackage[ibycus,polutonikogreek,english,french,ngerman]{babel}
\usepackage[autostyle]{csquotes}
%\usepackage{ngerman}
\usepackage{graphicx}
\usepackage[a4,center,cross]{crop}

\usepackage{listings}
\lstset{inputencoding=utf8,
  language=[LaTeX]{TeX},
  numbers=left, 
  %stepnumber=3, 
  numberfirstline=false,  
  numberstyle=\tiny\textsf,
  basicstyle=\footnotesize\ttfamily,
  %frame=tlrb,
  breaklines=true,
  postbreak=\mbox{$\hookrightarrow$},
  %showstringspace=false, 
  captionpos=b,
  literate={ä}{{\"a}}1{ö}{{\"o}}1{ü}{{\"u}}1{ß}{{\ss}}1
  }

\usepackage{showexpl}

\usepackage{beramono}

%\usepackage{euler}
\usepackage{yfonts}

\usepackage{textcomp}
\usepackage{textgreek}

\usepackage[pagewise]{lineno}

\usepackage{sidenotes}

\usepackage{multicol}

\usepackage{forest}

\usepackage{covington}
\renewcommand{\it}{}
\renewcommand{\rm}{}

\usepackage[hang]{footmisc}

\usepackage{parallel}

\usepackage{cjhebrew}

\usepackage{runic}

\usepackage{hieroglf}

\usepackage{hologo}

\usepackage{imakeidx}
\indexsetup{level= \subsection*, toclevel = subsection, noclearpage}

\makeindex[                title = Allgemeiner Index]
\makeindex[name = pakete,  title = Verzeichnis der Paketnamen]

\newcommand{\paket}[1]{\textsf{#1}\index[pakete]{#1}}


\usepackage[style=verbose-inote,pageref=true]{biblatex}
%\usepackage[style=historische-zeitschrift]{biblatex}
\addbibresource{lfgw-bibliographie.bib}
\defbibheading{bibliography}{\chapter{#1}}


\begin{document}
\setlength{\footnotemargin}{4mm}

\extratitle{\LaTeX{} für Geisteswissenschaftler
\newpage

\thispagestyle{empty}
\centering{\includegraphics[width=.85\textwidth]{aristoteles_de_anima}}
\medskip

\centering{Eine Passage aus Aristoteles' De Anima in einer mittelalterlichen Handschrift mit 
Interlinear- und Marginalglossen}
}

%\subject{edition dante}

\title{\LaTeX{} für Geisteswissenschaftler}
\subtitle{Ein Überblick über die wichtigsten Hilfsmittel für die Arbeit mit 
Texten als Forschungsgegenstand}

\author{Thomas Hilarius Meyer}
\date{}
\publishers{dante e.V.}

\lowertitleback{\copyright{} 2016 Thomas Hilarius Meyer
	Satz: thm mit \LaTeXe{} und \KOMAScript \\
        Haftungsausschluss: }

\maketitle

\addchap{Vorwort}

Philologie bedeutet

also müssten philologen besondere sorgfalt und liebe zur erstellung ...

Dieses Skript richtet sich an zwei Zielgruppen, deren Bedürfnisse verwandt, aber doch verschieden sind:

\begin{itemize}
 \item Angehörige der Geistes- und Sozialwissenschaften, die \LaTeX\ im Rahmen ihrer Arbeit
 -- sei es zur Erstellung einer Seminararbeit, einer Bachelor- oder Masterarbeit, eines
 Dissertationsprojektes oder eines komplexen Editionsvorhabens --
 einsetzen möchten und keine oder nur sehr geringe Kenntnisse des \LaTeX -Systems haben.
 \item Erfahrene \TeX -niker
\end{itemize}

Am Beginn dieses Buches muss ich den Leser um Vergebung für eine eigentlich unverzeihliche begriffliche
Unsauberkeit bitten: nichtmathematische Geisteswisschenschaftler!

Latex wird von einer sehr aktiven Community weiterentwickelt.

DTK verzeichnet ca. 50 NEUE Pakete im Quartal.

Deshalb große unübersichtlichkeit: Neulinge finden oft auch veraltetes bzw,. sehen vor lauter
Wald die Bäume nicht.

Deshalb Idee dieses Buches: Für die häufigsten Bedürfnisse im geisteswissenschaftlichen
Bereich gangbare, aktuelle Lösungen vorstellen.

Nähere Infos zur Intensiveren Nutzung enthalten immer die Paketdokumentationen.

Herzlichen Dank 

\tableofcontents

\chapter{Grundsätzliches}

Der Einsteiger in die Welt von \TeX\ und \LaTeX\ wird mit einer Vielzahl von Programmnamen und 
Spezialbegriffen, die zunächst einmal Verwirrung stiftet. 
Wie so oft hilft ein Blick in die Geschichte, die Vielfalt der Gegenwart zu verstehen...


\section{Zur Geschichte von \TeX{}, \LaTeX{} und co. -- zugleich eine Begriffsklärung}

Am Anfang der gesamten Entwicklung steht Donald Knuth, ein US-amerkianischer Mathematiker und 
Informatiker, der mit der typographischen Qualität zeitgenössischer mathematischer Texte 
unzufrieden war:

\begin{quote}
 Mathematics books and journals do not look as beautiful as they used to.
 It is not that mathemathical content is unsatisfactory, rather than the old and 
 well-developed traditions of typesetting have become too expensive.
 Fortunately it now appears that mathematics itself can be used to solve this problem.%
 \footcite[Zitiert nach: ][S. 1]{voss:einfuehrung}
\end{quote}

Die Antwort von Donald Knuth auf das von ihm wahrgenommene ästhetische Defizit war das  
programm \TeX . Als Ergänzung entwickelte er das Metafont -System zur Definition von 
Zeichensätzen.
\TeX bildet bis heute, die Grundlage des Gesamtsystems, doch wurde das Programm immer 
wieder erweitert und angepasst.

Die wichtigste Erweiterung erfolgte durch Leslie Lamport der eine ganze Reihe von 
inhaltlich ausgerichteten Befehlen auf \TeX -Basis definierte: \LaTeX\ war geboren.
Die bislang letzte Version dieses Programmes -- \LaTeXe\ -- stammt aus dem Jahre 1994
und definiert das Grundgerüst.

Zwei Dinge haben sich seither weiterentwickelt:

Das Ausgabeformat PDF hat sich als universelles Dokumentformat durchgesetzt; es wird von neueren
\LaTeX -Entwicklungen direkt erzeugt. (\TeX\ und das ursprüngliche \LaTeX\ erzeugten DVI-Dateien,
die man zu PS-Dateien konvertieren konnte, die man zu PDF-Dateien konvertieren konnte.)
Das ausführende Programm hieß nun pdflatex. 
Selbstverständlich versteht es auch Dateien in reinem \TeX ;
es kann per Kommandozeilenoption dazu gebracht werden, DVI-Dateien zu erzeugen, wenn dies aus 
bestimmten Gründen gewünscht ist.
Ruft man den Kommandozeilenbefehl \lstinline/latex/ auf, wird in aller Regel das Programm 
pdflatex gestartet.

Die zweite große Änderung betrifft das Eingabeformat:
\TeX\ und \LaTeX\ wurden so umgebaut, dass sie umittelbar Unicode-Zeichen verstehen.
Das ist v.a. für Sprachwissenschaftler interessant, stehen doch jetzt praktisch alle Schriftzeichen
der Welt unmittelbar zur Verfügung. 
Man kann das traditionelle pdflatex-Programm m.H. eines sog. Paketes dazu bringen, Unicode-Zeichen
zu verstehen
(Die entsprechenden Befehle heißen: \lstinline/usepackage[utf8]{inputenc}/ 
und \lstinline/usepackage[T1]{fontenc}/.)
oder das speziell für die Unicode-Untersützung entwickelte XeLaTeX verwenden.

Die neueste Entwicklung in der Welt der \TeX -Programme ist die Integration der Skriptsprache
Lua in die \LaTeX -engine: luaLaTeX wird als der kommende Standard gehandelt.

Die Beispiele des folgenden Buches sowie das Buch selbst sind -- wegen der weiten Verbreitung --
mit pdflatex erzeugt worden.

\section{Arbeitswerkzeuge}

empfehlenswert für die tägliche Arbeit mit \LaTeX gerade bei größeren Projekten ist eine IDE.
Diese Schrift wurde mit kile erstellt.
Alternative: overleaf oder sharelatex.
Doch zuerst sollte man die Einzelwerkzeuge kennen, die in solchen \enquote{Meta-Werkzeugen}
zusammengefasst sind.

\minisec{Einzelwerkzeuge}

\LaTeX -Dokumente lassen sich mit jedem beliebigen \textbf{Editor} erstellen.
Doch es gibt ein paar Qualitätsmerkmale, auf die man bei der Auswahl achten sollte, weil
sie die tägliche Arbeit wesentlich erleichtern:

Syntax-Highlighting hebt \LaTeX -Befehle sowie bestimmte Inhalte wie Überschriften etc. farblich
hervor, was die Übersicht über das Dokument wesentlich erleichtert. Auch die Navigation in
den manchmal Geschachtelten geschweiften Klammern kann durch einen guten Editor wesentlich
erleichtert werden: Fährt man eine schließende Klammer an, wird die jeweils dazugehörende
öffnende Klammer hervorgehoben.

Das zweite Merkmal, das ein Editor (v.a. für den Einsatz in geisteswissenschaftlichen Projekten!) 
erfüllen muss, ist der Umgang mit Unicode. Moderne Linux-Distributionen verwenden Unicode (utf8)
bereits standardmäßig als Codierung, manchmal muss das manuell eingestellt werden.

Einige empfehlenswerte Editoren für die Arbeit mit \LaTeX\ sind:
für Linux: kwrite,  und kate (beide aus dem KDE-Projekt) --
für Windows: Notepad++.

Einen Sonderfall stellt der plattformübergreifend verfügbare Editor Emacs mit Auc-Tex-Ergänzung
dar: Eigentlich handelt es sich dabei bereits um eine ausgewachsene Entwicklungsumgebung (s.u.);
Emacs-Anhänger erledigen noch viele andere Dinge mit \enquote{ihrem} Editor, doch ist die 
Einarbeitung für Uneingeweihte etwas gewöhnungsbedürftig.
Der Autor dieser Schrift war nie geekish genug für Emacs...

Kommandozeile

pdflatex

biber


\minisec{Integrierte Entwicklungsumgebungen (IDE)}

kile (Linux)

texworks (Win + Linux)

Texshop (Mac)

TexnicCenter (Windows)


\minisec{online-Lösungen}
\texttt{www.overleaf.com} \index{overleaf} und \texttt{www.sharelatex.com} \index{sharelatex}
\footcite{meyer:dtk2015/1}



\section{Erste Schritte in \LaTeX}

Anlegen einer Datei

Dokumentsrtuktur: Präambel, Dokumentklasse etc.

Übersetzen der \LaTeX{}-Datei

Betrachten und evtl. Ausdrucken der PDF-Datei


\section{Musterprojekt: eine beispielhafte Hausarbeit}

Angenommen, wir sollen eine Hausarbeit schreiben.
Wir wollen Grafiken einbauen, unsere Literatur automatisch verarbeiten und ein Register erstellen.
Außerdem soll die Typographie schön sein.

Also brauchen wir für den Anfang erst einmal eine Datei mit ein paar Zauberworten:


\minisec{Datei: hausarbeit.tex}

\begin{lstlisting}
\documentclass{scrreprt}
 
\usepackage[utf8]{inputenc}
\usepackage[T1]{fontenc}

\usepackage[ngerman]{babel}
\usepackage{csquotes}
\usepackage{graphicx}

\usepackage{imakeidx}

\usepackage[style=verbose-inote,pageref=true]{biblatex}
\addbibresource{meine-bibliographie.bib}


\begin{document}

\title{\LaTeX{} und der Sinn des Lebens}
\subtitle{Eine vielosofische Hausarbeit}

\author{Eberhard Knesenbeck}
\date{[Stand: \today]}

\maketitle

\tableofcontents

blubb, blah...

\printbibliography
\end{document}
\end{lstlisting}

Was bedeuten die Anweisungen im einzelnen?

\begin{description}
 \item[1] Das Einstellen der Dokumentklasse definiert das grundsätzliche Aussehen des 
 Textes: geht es um ein Buch, einen Artikel oder - wie hier - eine Hausarbeit. 
 Die Angabe \lstinline/scrrprt/ bedeutet, dass ein ``Report'' (also z.B. eine Hausarbeit)
 m.H. der Definitionen des \KOMAScript -Paketes von Markus Kohm erstellt werden soll.
 (Mehr dazu vgl. \pageref{komaskript}
 \item[3-4] Zwei Zauberzeilen, die dazu führen, dass alle möglichen Unicode-Sonderzeichen
 benutzt werden dürfen und auch ausgegeben werden können wie z.B. ä, ö, ü, ß, æ, ï oder þ.
 (Achtung! Der Editor muss auf Unicode eingestellt sein! - vgl. S. \pageref{unicode}
 \item[6-7] \LaTeX\ benutzt jetzt deutsche Begriffe (Bibliographie statt Bibliography) und 
 Silbentrennmuster (\paket{babel}). 
 \footnote{Etwas genauer: Es gelten die Regeln der sog. Rechtschreibreform von 1996.
 Sollen die Regeln von 1901 gelten, muss als Option statt dessen \lstinline/german/ angeben
 werden. Vgl. \ref{babel} auf S. \pageref{babel}}
 Außerdem werden (Z. 7) deutsche Gepflogenheiten bei den Anführungszeichen
 verwendet (\paket{csquotes}). 
 Die beiden zaubersprüche sollten in keiner deutschen datei fehlen.
 \item[8] Das Paket \paket{graphicx} ist ein mächtiges Werkzeug um Grafiken einzubinden.
 Wir laden es vorsorglich; wenn unser Text nicht lang genug ist, können wir Bilder dazunehmen...
 \item[10] Das paket \paket{imakeidx} wird uns später erlauben, sehr einfach ein Register zu unserer
 Arbeit auszugeben.
 Für komplexere Aufgaben wie Orts- und Personenregister, Bibelstellenregister etc. lässt sich
 imakeidx umfangreich konfigurieren.
 \item[12-13] Diese beiden recht abschreckenden Zauberformeln werden uns erlauben, unsere
 gesamte Literaturverwaltung mit allen Formalien automatisch erledigen zu lassen.
 Hier sagen wir \LaTeX\ , dass wir unsere Bibliographiedaten in einer Datei namens 
 meine-bibliographie.bib aufheben, dass das Programm biblatex für uns die Arbeit machen soll
 und dass es sich dabei an einem bestimmten, eher traditionellen geisteswissenschaftlichen
 Stil ausrichten soll.
 \item[15] Jetzt wird es ernst: Unser eigentliches Dokument beginnt.
 \item[18-19] Erst mal legen wir einen guten Titel und Untertitel fest...
 \item[21] ... und sagen, wer wir sind.
 \item[22] Diese Zeile sorgt dafür, dass auf dem Titelblatt das jeweils aktuelle Datum als 
 Stand der Arbeit ausgegeben wird. Das kann praktisch sein, wenn man seltener seinen Schreibtisch
 aufräumt, als neue Fassungen ausdruckt.\\
 Achtung! Soll (z.B. in der Endfassung) kein Datum ausgegeben werden, so muss man explizit
 \lstinline/\date{}/ angeben. Löscht man die \lstinline/\date/-Anweisung ganz, wird 
 automatisch das heutige Datum ausgegeben!
 \item[24] Achtung! Erst mit dem Befehl \lstinline/\maketitle/ wird \LaTeX\ dazu gebracht,
 die definierten Titelseiten auch wirklich \emph{auszugeben}!
 \item[26] Dieser Befehl gibt das Inhaltsverzeichnis aus.
 \item[28] Jetzt kommen endlich unsere wertvollen Inhalte.
 \item[30] Hier wird das automatisch erstellte Inhaltsverzeichnis ausgegeben.
 \item[31] Diese letzte Zeile schließt \LaTeX -Dokumente ab.
 \end{description}
 
 
\minisec{Datei: meine-bibliographie.bib}

Damit sich die Datei kompilieren lässt, muss die in Zeile 13 angekündigte Datei mit 
bibliografischen Angaben auch existieren!

Sie habe einen einzigen Eintrag mit einem wichtigen Buch für \LaTeX -Einsteiger.
Legen wir sie an:

\begin{lstlisting}
@Book{voss:einfuehrung,
 author = {Heinrich Voß}, 
 title = {Einführung in \LaTeX},
 publisher = {dante e.V. and Lehmanns Media},
 location = {Berlin and Heidelberg},
 year = {2016},
 edition = {2. Aufl},
}
\end{lstlisting}

Die bibliografien-Datei unseres Projektes enthält also einen einzigen Eintrag:
\bigskip

\framebox[75mm]{
  \parbox{65mm}{
    \cite{voss:einfuehrung}
  }
}
\parbox{40mm}{
   \emph{übrigens sehr lesenswert...}
}
\bigskip

Dieser Titel ist übrigens automatisch aufgelöst worden; 
in der Datei steht nur \lstinline/\cite{voss:einfuehrung}/.
Allein diese Funktionalität -- und all die Tricks, die sich in dem Zusammenhang anstellen
lassen, rechtfertigen den Einsatz von \LaTeX\ und co. (Mehr dazu S. \pageref{biblatex} ff.)


%\chapter{Typographische Gestaltung}

\chapter{Texte schreiben}

Geistes- und humanwissenschaftliche Arbeiten haben besondere Anforderungen an die typografische
Gestaltung. 
Die häufigsten Anforderungen werden im folgenden der Reihe nach durchgegangen.
Maches davon ist \LaTeX -Standard; anderes wird außerhalb der Geisteswissenschaften eher
selten gebraucht.
Ein großer Bereich, für den \LaTeX\ eigentlich bekannt ist, bleibt ganz außen vor:
der Satz von komplexen mathematischen Formeln und Gleichungen.


\section{Vorüberlegungen zu typographischer Schönheit und Funktionalität}

\dictum[Louis Sullivan]{Form follows fuction.}

\begin{quote}
 Das Selbermachen ist längst üblich, die Ergebnisse sind oft fragwürdig,
 weil die Laien-Typografen nicht sehen, was nicht stimmt und nicht wissen
 können, worauf es ankommt.
 So gewöhnt man sich an falsche und schlechte Typografie.
 \footcite[S. 9]{erste_hilfe}
 \end{quote}
 
\begin{quotation}
 \emph{Du warst doch mal Sekretärin.
 Ich muß meine Diss schreiben, kannst Du mir zeigen, wie das geht?}
 
 \emph{Was} ein Doktorand oder Diplomand zu schreiben hat, das hat er studiert.
 \emph{Wie} er es zu schreiben hat, hat er nicht studiert, er setzt irgendwie drauflos.
 Mit unübersichtlicher und schlecht lesbarer Laien-Typografie schadet er oft genug seiner Arbeit.
 \footcite[S. 86]{erste_hilfe}
\end{quotation}


Längeres Zitat aus Erste Hilfe Typographie o.ä.


\section{Wahl der Dokumentklasse}

In der ersten Zeile eines \LaTeX -Dokumentes wird die Dokumentklasse des folgenden Textes 
festgelegt. Die Dokumentklasse definiert die grundsätzlichen Spielregeln:
Welche Gliederungsebenen sind vorgesehen, wie ist die grundsätzliche Formatierung aufgebaut etc.

Es existiert eine sehr große Vielzahl von verschiedenen Dokumentklassen;
so gibt es für Hausarbeiten oder Dissertationen zahlreicher Universitäten eine eigene
Dokumentklasse. Existiert eine solche, sollte man sie i.d.R. auch benutzen.

Die folgende Einführung beschränkt sich auf die Klassen des \KOMAScript -Paketes von
Markus Kohm, weil diese eine überragende typographische Qualität sowie zahllose Möglichkeiten
zur individuellen Anpassung bieten.
\footnote{Wer allerdings ihre volle Funktionalität nutzen will, sollte sich die originale 
Dokumentation beschaffen und sich mit ihrer Hilfe genauer einarbeiten...}
Hinzu treten zwei Dokumentklassen für speziellere Fälle, nämlich für Präsentationen und 
Prüfungen.

\begin{description}
 \item[scrartcl] ist die \KOMAScript -Klasse für Artikel.
 \item[scrreprt] kann gut für längere Seminar- oder auch Bachelor-Arbeiten genutzt werden.
 \item[scrbook] ist die Klasse für Bücher mit zahlreichen ausgereiften Funktionen für
  die Titelei etc.
 \item[exam] ist eine eigene Klasse zur Entwicklung von Aufgabenblättern zu Prüfungszwecken.
 \item[beamer] stellt eine ganze Reihe von features für die Erstellung von Präsentationen
  zur Verfügung. Mit etwas Einarbeitung gelingen damit schneller bessere Präsentationen als 
  mit der verbreiteten Konkurrenz.
\end{description}


\minisec{Zum Unterschied zwischen KLASSE und Paket}

Eine \textbf{Dokumentklasse} ...

\textbf{Pakete} ...


\section{Nationale Besonderheiten - das Paket babel}
\label{babel}

\dictum[HPW 104]{Wenn ich nur das Wort Europa höre, entsichere ich meine Dicke Bertha.}

\paket{babel}

Aufruf: \lstinline/\usepackage[Liste mit benutzen Sprachen]{babel}/ 

Babel unterstützt eine ganze Reihe von Sprachen; für Geisteswisschenschaftler am wichtigsten sind
wohl:

\begin{description}
 \item[german] Deutsch, alte Rechtschreibung (1901)
 \item[ngerman] Deutsch gemäß der Rechtschreibreform von 1996
 \item[greek] Neugriechisch
 \item[polutonikogreek] (XXX was ist der Unterschied)
\end{description}

Die als letztes genannte Sprache stellt die Standardsprache des Dokuments (die am Anfang
eingestellt ist) dar.

Auch der Dokumentklasse sollte man die Standardsprache des Dokumentes bekanntgeben.
Denn zahlreiche Pakete , die später geladen werden, werten diese Sprachoption aus und 
passen sich an (z.B. \paket{varioref}).

Somit ergibt sich folgender Typische Dokumentbeginn:

\begin{lstlisting}
 \documentclass[ngerman]{scrreprt}
 \usepackage[polutonikogreek,ngerman]{babel}
\end{lstlisting}

\minisec{Wechsel zu einer anderen Sprache}

Im laufenden Dokument kann auf eine der im Paketaufruf angegebenen Sprachen umgeschaltet werden:

\begin{lstlisting} 
 \selectlanguage{polutonikogreek}
\end{lstlisting}



\minisec{Alternative: Das Paket polyglossia}

\paket{polyglossia}


\section{Seitengestaltung und Seitenspiegel}
\label{komaskript}
\index{Seitenspiegel} \index{Goldener Schnitt}

Die folgenden Angaben sind nur von Interesse, wenn man als Papierformat nicht Din-A-4-Papier 
nutzen will.

\KOMAScript{}


\minisec{Einstellen des Papierformats}

\paket{geometry}
\index{Papierformat}


\minisec{Beschnittmarken}
\index{Beschnittmarken}
\paket{crop}


\minisec{Ein- oder zweiseitiges Layout}


\section{Textgliederung}

\begin{lstlisting}
 \part{Ein Teil des Teils}
 
 \chapter{Ein Kapitel}
 
 \section{Ein Abschnitt}
 
 \subsection{Ein Unterabschnitt}
 
 \subsubsection{Ein kleiner Unterabschnitt}
 
 \chapter*{Ein Kapitel ohne Eintrag im Inhaltsverzeichnis}
 
 \section*{Ein Abschnitt ohne Eintr. im Inhaltsverz.}
 
 \subsection*{Ein Unterabs. o. E. im Iv.}
 
 \subsubsection*{Ein kleiner Unterabschnitt o.E.i.I.}
 
 \chapter[Kurzfssg. f. d. Kolumnentitel]{Ein Kapitel}
 
 \section[Kurzfssg. f. d. Kolumnentitel]{Ein Abschnitt}
 
 \subsection[Kurzfssg. f. d. Kolumnentitel]{Ein Unterabschnitt}
\end{lstlisting}

Texte lassen sich in \LaTeX\ m.H. eines hierarchischen Systems von Überschriften gliedern:

\begin{description}
 \item[1-9] Texte der Kategorien 
    \lstinline/book/, 
    \lstinline/scrbook/, 
    \lstinline/report/ und 
    \lstinline/scrrprt/ 
 bestehen aus Teilen (\lstinline/\part{}/)
 \footnote{Ehrlich gesagt: eher selten. 
 \lstinline/part/ 
 erzeugt einen sog. ``Zwischentitel'',
 der im deutschsprachigen Raum eher unüblich ist. Ausprobieren! }
 und Kapitel (\lstinline/\chapter/);
 desweiteren lassen sich alle Texte in Abschnitte und Unterabschnitte untergliedern 
 (\lstinline/\section{}/ bis \lstinline/\subsubsection{}/) 
 gliedern.
 Jede Überschrift beeinflusst die Kolumnentitel (s.u.) und erzeugt automatisch einen Eintrag
 im Inhaltsverzeichnis.
 \item[1-9] Die sog. ``Sternvarianten'' der Gliederungsbefehle erzeugen die gleiche
 Formatierung der Überschriften, hinterlassen aber keinen Eintrag im Inhaltsverzeichnis.
 Außerdem werden die Überschriften nicht numeriert.
 \item[19-23] Die Kapitel und ggf. Abschnitssüberschriften werden auch als lebende Kolumnentitel
 verwendet. Wenn sie dafür zu lang sind, kann man in eckigen Klammern einen alternativen
 Kurztitel zur Verwendung als Kolumnentitel angeben. \index{Kolumnentitel}
\end{description}



\section{Schriftauszeichnungen}
\index{Auzeichnungsschriften}

\dictum[xxx]{Weniger ist mehr.}

\minisec{Auszeichnung über Schriftschnitt}
\index{Schriftschnitt}
\index{kursiv}
\index{Kapitälchen}
\index{fett}
\index{Unterstreichung}

\begin{tabular}{lll}
 \lstinline/\emph{}/ 		&	\emph{kursiv} 		&	+ \\
 \lstinline/\textsl{}/		&	\textsl{schräggestellt} &	-- -- \\
 \lstinline/\textsc{}/		&	\textsc{Kapitälchen}	&	+ \\
 \lstinline/\textbf{}/ 		&	\textbf{fett} 		&	+/- \\
 \lstinline/\underline{}/ 	&	\underline{unterstrichen} &	-- \\
 \lstinline/\texttt{}/		&	\texttt{Schreibmaschinenschrift} &	? \\ 
 \end{tabular} 


\minisec{Veränderung der Schriftgröße}
\index{Schriftgröße}

\begin{tabular}{ll}
 \lstinline/\Huge{}/		&	\Huge{Riesig!} \\
 \lstinline/\huge{}/		&	\Huge{riesig} \\
 \lstinline/\LARGE{}/		&	\LARGE{sehr, sehr groß} \\
 \lstinline/\Large{}/		&	\Large{sehr groß} \\
 \lstinline/\large{}/		&	\large{groß} \\
 \lstinline/\normalsize{}/	&	\normalsize{normal groß (Grundschriftgröße)} \\
 \lstinline/\small{}/		&	\small{klein} \\
 \lstinline/\footnotesize{}/	&	\footnotesize{so groß wie die Fußnoten} \\
 \lstinline/\scriptsize{}/	&	\scriptsize{für Kleingedrucktes} \\
 \lstinline/\tiny{}/		&	\tiny{wirklich winzig} \\
\end{tabular}


\section{Besondere Schriftarten}

Am bequemsten lassen sich in \LaTeX\ Schriften verwenden, für die bereits ein fertiges
Paket existiert, dass lediglich in der Präambel eingebunden werden muss.


\minisec{Alte deutsche Schriften: Fraktur-Schriften}

{\frakfamily
Mit \LaTeX\ lassen sich drei ältere deutsche Schriftarten sehr komfortabel und mit 
professionellem Anspruch verwenden:}

\begin{enumerate}
 \item {\frakfamily Die Frakturschrift, die in Deutschland bis: 1941 die Standardschrift war.
    Sie wirkt sehr feingliedrig, ist gut zu lesen (mit etwas: Übung) und benötigt 
    übrigens: sehr wenig Platz.}
 \item {\gothfamily Die gotische Schrift, die direkt auf Johannes: Gutenberg zurückgeht.}
 \item {\swabfamily Die Schwabacher, die vor allem in der frühen Neuzeit eine sehr weit
    verbreitete Gebrauchsschrift war.}
\end{enumerate}

{\frakfamily
Alle drei Schriftarten werden durch das: Paket}
\paket{yfonts}
{\frakfamily
von Yannis Haralambous: zugänglich gemacht.

(Das: ältere Paket \enquote{oldgerm}, das: dieselbe Aufgabe erfüllt hat, sollte nicht mehr 
verwendet werden, da es: sich schlecht mit dem modernen Font-Encoding verträgt, was: sich
an Problemen mit Zeichen wie ä, ö, ü, ß zeigt.)

Abweichende Schriftschnitte wie kursiv, geneigt oder (halb-)fett stehen nicht zur 
Verfügung.

Doch dabei lauert eine Stolperfalle:
Das: Paket bietet nur die Unterstützung für die Fonts:, das: heißt zahlreiche Erleichterungen
zur Eingabe etc., enthält aber nicht selbst die Fonts:.
Diese müssen separat installiert werden, sonst lässt sich}
\paket{yfonts}
{\frakfamily 
zwar einbinden, doch der Aufruf der Schriften verursacht eine Fehlermeldung und 
statt hübscher Frakturschrift stehen in der PDF-Datei nur leere Flächen.

Das: Paket stellt sechs: Befehle zur Verfügung:}

\begin{itemize}
 \item \lstinline/\frakfamily/
 \item \lstinline/\textfrak{...}/
 \item \lstinline/\gothfamily/
 \item \lstinline/\textgoth{...}/
 \item \lstinline/\swabfamily/
 \item \lstinline/\textswab{...}/
\end{itemize}

{\frakfamily
Der jeweils: erste Befehl stellt die Schrift dauerhaft auf Fraktur, Gotisch oder
Schwabacher um; der jeweils: zweite Befehl dient, um eine kürzere Passage in der 
jeweiligen Schriftart aus:zugeben.}

{\frakfamily 
Typographisch muß man beachten, daß die Frakturschrift zwei Zeichen für das: kleine S hat:
}

\begin{itemize}
 \item \textfrak{s: steht am Wortausgang sowie auch bei zusammengesetzten Wörtern.
  Es: wird als:}
  \lstinline/s:/  
  \textfrak{eingegeben.}
 \item \textfrak{s steht im Wortinneren.}
\end{itemize}

{\frakfamily
Wenn man schon Fraktur-Schrift benutzt, sollte man sich ein wenig aus:kennen und muß
sich in den Schriftcharakter einfühlen. So verträgt sich Fraktur mit der Rechtschreibreform
von 1996 nur sehr schlecht: Wörter wie \enquote{daß} verlangen nach der SZ-Ligatur.
Die Paketdokumentation enthält nicht nur Hinweise zur technischen Benutzung des: 
Paketes:, sondern auch zu gestalterischen Fragen im Zusammenhang mit der Frakturschrift.
}


\minisec{Schreibschriften}
\index{Schreibschrift}

Das Paket \paket{schulschriften} von Walter Entenmann bietet verschiedene deutsche Schulschriften.
\footcite[Zur Geschichte der deutschen Schulschriften sowie dem Vorgehen zu ihrer Implementierung 
in \LaTeX :][]{entenmann:dtk2012/4}

Zu beachten ist, dass das Paket nicht mit \lstinline/\usepackage/ eingebunden werden muss;
das ist auch nicht möglich, da es keine Paketdatei (\enquote{schulschriften.sty}) anbietet.
Es reicht aus, die Fontdateien zu installieren.

Die Fonts müssen dann über den eigentlichen Fontauswahlmechanismus von \LaTeXe\ angesprochen werden:

Sütterlin-Schrift mit gerader Feder:

\begin{LTXexample}
{\usefont{T1}{wesu}{m}{n}
\huge
Ich kann schreiben!}
\end{LTXexample}

Sütterlin-Schrift geneigt mit Bandzugfeder:

\begin{LTXexample}
{\usefont{T1}{wesu}{b}{sl}
\huge
Ich kann schreiben!}
\end{LTXexample}


Deutsche Normalschrift:

\begin{LTXexample}
{\usefont{T1}{wedn}{m}{sl}
\huge
Ich kann schreiben!}
\end{LTXexample}

Lateinische Ausgangsschrift:

\begin{LTXexample}
{\usefont{T1}{wela}{m}{sl}
\huge
Ich kann schreiben!}
\end{LTXexample}

Schulausgangsschrift:

\begin{LTXexample}
{\usefont{T1}{wesa}{m}{sl}
\huge
Ich kann schreiben!}
\end{LTXexample}

Vereinfachte Ausgangsschrift:

\begin{LTXexample}
{\usefont{T1}{weva}{m}{sl}
\huge
Ich kann schreiben!}
\end{LTXexample}

Die ausgezeichnete deutschsprachige Paketdokumentation eignet sich übrigens sehr gut,
den dahinterliegenden Fontauswahlmechanismus von \LaTeX\ kennenzulernen.

In der Praxis wird man die Fontaufrufe in ein selbstdefiniertes Makro verpacken, 
vgl. \ref{makros} auf Seite \pageref{makros}.


\section{Sonderzeichen}

\cite{voss:dtk20011/1}

\minisec{Der Wortzwischenraum}

Der übliche Wortzwischenraum (Fachbezeichnung: Spatium) wird von \LaTeX\ beim Absatzumbruch innerhalb einer Zeile
gleichmäßig verteilt. Die Eingabe erfolt in Form von normalen Leerzeichen oder als 
einfacher Zeilenwechsel.

Folgende beiden Eingaben sind also gleichwertig:

\begin{LTXexample}
 superbia avaritia invidia ZORN UNKEUSCHHEIT UNMAESSIGKEIT acedia
 
 Wort1
 Wort2
 Wort3
 Wort4
 Wort5
 Wort6
 Wort7
\end{LTXexample}

Ein sog. \enquote{fester Ausschuss} ist in der Setzersprache ein Wortzwischenraum, an dem 
kein Zeilenwechsel durchgeführt werden darf. Er wird mit dem Zeichen \lstinline/~/
codiert:

\begin{LTXexample}
 Kunst und Kultur im 19.~Jahrhundert
\end{LTXexample}

Manchmal ist es nötig, an einzelnen Stellen längere Spatien zu werden, etwa um Halbverse 
in epischen Dichtungen zu markieren. Dazu kann man mehrere feste Ausschüsse verbinden:

\begin{lstlisting}
 \begin{verse}
 Uns ist in alten maeren ~ wunders vil geseit\\
 von heleden lobebaeren ~ von von grozer arebeit\\
 von fröuden, hochgeziten, ~ von weinen und von klagen,\\
 von küener recken striten ~ muget ir nu wunders hoeren sagen.
 \end{verse}
\end{lstlisting}

Daraus wird:

 \begin{verse}
 Uns ist in alten maeren ~ wunders vil geseit\\
 von heleden lobebaeren ~ von von grozer arebeit\\
 von fröuden, hochgeziten, ~ von weinen und von klagen,\\
 von küener recken striten ~ muget ir nu wunders hoeren sagen.
 \end{verse}


\minisec{Leerzeichen vermeiden -- das Zeichen \%}

\LaTeX\ hält sich bekanntlich nicht an die Zeilenaufteilung der Quelldatei. Erst eine Leerzeile
wird als Absatzgrenze interpretiert. Einfache Zeilenwechsel führen zu einem normalen 
Wortzwischenraum. Dies kann unerwünscht sein:

Besonders bei komplexeren \LaTeX -Strukturen kann es sinnvoll sein, Zeilenwechsel einzufügen,
z.B. zwischen einzelnen Befehlen oder einem Wort und einer folgenden Fußnote.
An diesen Stellen soll aber kein Wortzischenraum -- womöglich sogar ein sich ergebender 
Zeilenwechsel -- eingefügt werden.

Dazu kann man das Kommentarzeichen \% an das Zeilenende setzen. Der Zeilenwechsel wird auf diese 
Weise quasi \enquote{auskommentiert}, bei der Übersetzung des Dokuments mit \LaTeX\
verschwindet er vollständig.

Man beachte den Unterschied:

\begin{LTXexample}
 Wort1
 Wort2
 
 Wort3%
 Wort4
\end{LTXexample}


\minisec{Sonderzeichen für Germanisten}

Quasi unterhalb der Ebene von Unicode mit seinem Zugang zum gesamten Zeichensatz hält 
\LaTeX\ einige Sonderzeichen bereit, die z.B. für Germanisten von besonderem Interesse
sein dürften. Sie können als besondere Befehle eingegeben werden:

\begin{tabular}{ll}
 kleines \th & 	\lstinline/\th/ \\
 großes \TH & 	\lstinline/\TH/ \\
\end{tabular}
\index{Thorn}

Lädt man zusätzlich das \paket{textcomp}, stehen eine ganze Reihe weiterer Symbole
zur Verfügung:

\begin{tabular}{ll}
 \pounds & 	\lstinline/\pounds/ \\
 \copyright & 	\lstinline/\copyright/ \\
 \texteuro & 	\lstinline/\texteuro/ \\
 \textborn & 	\lstinline/\textborn/ \\
 \textmarried & 	\lstinline/\textmarried/ \\
 \textdied & 	\lstinline/\textdied/ \\
 \end{tabular}
 
Dies stellt nur eine winzige Teilmenge der in \LaTeX\ zur Verfügung stehenden Sonderzeichen dar;
eine vollständige Übersicht bietet die \enquote{Comprehensive \LaTeX\ Symbol List} von
Scott Pakin (331 Seiten, Stand Nov. 2015).
\footnote{\lstinline/texdoc comprehensive/}


\minisec{Akzenze und Diakritika}

Diakritika \index{Diakritika},
Betonungszeichen etc.


\minisec{Unicode-Codes}

Was immer geht: 
vgl. Abschnitt \ref{unicodeeingabe} auf S. \pageref{unicodeeingabe} 
(für die Eingabemethode)
bzw. Abschnitt \ref{utf8codes} auf S. \pageref{utf8codes} 
(für eine Übersicht über häufig benötigte Unicode-Zeichen und ihre Codes).


\minisec{Einzelne griechische Buchstaben}
\label{griechEinzelbuchstaben}

Das Paket \paket{textgreek} erlaubt die einfache Eingabe griechischer Buchstaben.
Es ist natürlich weniger geeignet, ganze Texte auf Griechisch zu erfassen, kann aber sehr
praktisch sein, wenn es eben nur um wenige einzelne Zeichen geht:

\begin{center}
\begin{tabular}{llll}
 \textalpha & 	\lstinline/\textalpha/ &	\textAlpha &	\lstinline/\textAlpha/ \\
 \textbeta & 	\lstinline/\textbeta/ &		\textBeta &	\lstinline/\textBeta/ \\
 \textgamma & 	\lstinline/\textgamma/ &	\textGamma &	\lstinline/\textGamma/ \\
 \textdelta & 	\lstinline/\textdelta/ &	\textDelta &	\lstinline/\textDelta/ \\
 \textepsilon & \lstinline/\textepsilon/ &	\textEpsilon &	\lstinline/\textEpsilon/ \\
 \textzeta & 	\lstinline/\textzeta/ &		\textZeta &	\lstinline/\textZeta/ \\
 \texteta & 	\lstinline/\texteta/ &		\textEta &	\lstinline/\textEta/ \\
 \texttheta & 	\lstinline/\texttheta/ &	\textTheta &	\lstinline/\textTheta/ \\
 \textiota & 	\lstinline/\textiota/ &		\textIota &	\lstinline/\textIota/ \\
 \textkappa & 	\lstinline/\textkappa/ &	\textKappa &	\lstinline/\textKappa/ \\
 \textlambda & 	\lstinline/\textlambda/ &	\textLambda &	\lstinline/\textLambda/ \\
 \textmu & 	\lstinline/\textmu/ &		\textMu &	\lstinline/\textMu/ \\
 \textnu & 	\lstinline/\textnu/ &		\textNu &	\lstinline/\textNu/ \\
 \textxi & 	\lstinline/\textxi/ &		\textXi &	\lstinline/\textXi/ \\
 \textomikron & \lstinline/\textomikron/ &	\textOmikron &	\lstinline/\textOmikron/ \\
 \textpi & 	\lstinline/\textpi/ &		\textPi &	\lstinline/\textPi/ \\
 \textrho & 	\lstinline/\textrho/ &		\textRho &	\lstinline/\textRho/ \\
 \textsigma & 	\lstinline/\textsigma/ &	\textSigma &	\lstinline/\textSigma/ \\
 \texttau & 	\lstinline/\texttau/ &		\textTau &	\lstinline/\textTau/ \\
 \textupsilon & \lstinline/\textupsilon/ & 	\textUpsilon &	\lstinline/\textUpsilon/ \\
 \textphi & 	\lstinline/\textphi/ &		\textPhi &	\lstinline/\textPhi/ \\
 \textchi & 	\lstinline/\textchi/ &		\textChi &	\lstinline/\textChi/ \\
 \textpsi & 	\lstinline/\textpsi/ &		\textPsi &	\lstinline/\textPsi/ \\
 \textomega & 	\lstinline/\textomega/ &	\textOmega &	\lstinline/\textOmega/ \\
\end{tabular}
\end{center}

Offene Frage: Akzente

Wo liegt die Schmerzgrenze für dieses Verfahren?
Wahrscheinlich bei etwas über einem Wort...

Comprehensive S. 


\minisec{Ein \enquote{kleines Ärgernis}: Anführungszeichen}
\label{enquote}

Nicht einfach die Gänsefüßchen \lstinline/"/ benutzen, das diese nicht korrekt interpretiert
werden können: Es ist für den Computer schwer erkennbar, wo die Anführung beginnt und wo sie
endet. Entsprechende regelbasierte Automatiken sind notorisch fehleranfällig.
Da die zugrundeliegenden Regeln nicht ganz trivial sind,
\footnote{Lesetypographie seite ...}
kommen -- auch in professionell gesetzten Büchern häufig Fehler vor.
Man sollte daher möglichst auf eine händische Eingabe ganz verzichten.

Am sinnvollsten ist der Weg, das Paket \paket{csquotes} einzubinden (in der Präambel) und auf 
eine manuelle Eingabe der Anführungszeichen ganz zu verzichten: Statt dessen definiert das Paket
ein Makro namens \lstinline/\enquote{}/, das als Ersatz für die Anführungszeichen eingegeben
wird.

Das Paket erkennt, welche Sprache für das Dokument (mittels \paket{babel} oder \paket{polyglossia})
eingestellt worden ist, und ersetzt die codierung \lstinline/\enquote{}/ durch die 
jeweils korrekte Form -- auch bei geschachtelt auftretenden Anführungszeichen:

\begin{LTXexample}
Er sprach: \enquote{Hast du den \enquote{Faust} gelesen?} 

\selectlanguage{french}
Er sprach: \enquote{Hast du den \enquote{Faust} gelesen?} 

\selectlanguage{english}
Er sprach: \enquote{Hast du den \enquote{Faust} gelesen?} 
\end{LTXexample}

Die Verwendung von \paket{csquotes} hat noch einen zweiten Vorteil:
Man kann an einer Stelle, nämlich bei den Optionen des Paketes, auch andere Anführungszeichen 
einstellen und muss nicht alle einzelnen Stellen umcodieren.

Arbeitsumgebungen wie kile lassen sich übrigens so einstellen, dass die Eingabe von Shift + 2
automatisch \lstinline/\enquote{/ oder \lstinline/}/ erzeugt.


rotatebox

reflectbox


\section{Listen und Aufzählungen}

\LaTeX\ bietet eine ganze Reihe von Möglichkeiten, Listen und Aufzählungen zu gestalten.
Hier werden nur die grundlegenden Funktionen gezeigt; daneben gibt zahlreiche Möglichkeiten zur
Modifikation.\footcite[S. 295ff.]{voss:einfuehrung}


\minisec{``Triviale'' Listen ohne alles: trivlist}

Am anspruchlosesten ist die einfache Auflistung ohne irgendwelche Formatierung:

\begin{LTXexample}
\begin{trivlist}
 \item der erste Punkt,
 \item der zweite Punkt,
 \item der dritte Punkt.
\end{trivlist} 
\end{LTXexample}


\minisec{``Normale'' Liste mit Aufzählungszeichen: itemize}

\begin{LTXexample}
\begin{itemize}
 \item der erste Punkt,
 \item der zweite Punkt,
 \item der dritte Punkt.
\end{itemize} 
\end{LTXexample}

XXX Erklären: Verändern des Aufzählungszeichens für einen einzelnen Punkt und generell.


\minisec{Durchnumerierte Aufzählung: enumerate}

\begin{LTXexample}
\begin{enumerate}
 \item der erste Punkt,
 \item der zweite Punkt,
 \item der dritte Punkt.
\end{enumerate} 
\end{LTXexample}

Durch Verändern des Zählers \lstinline/enumi/ kann man den Startpunkt der 
Aufzählung verändern:

\begin{LTXexample}
\begin{enumerate}
 \setcounter{enumi}{123}
 \item der erste Punkt,
 \item der zweite Punkt,
 \item der dritte Punkt.
\end{enumerate} 
\end{LTXexample}


\minisec{Worterklärung, Glossar I: description}
\index{Glossar}

Besonders für Worterklärungen o.ä. eignet sich die Umgebung \lstinline/description/,
die zu jedem \lstinline/\item/  eine Angabe in eckigen Klammern erwartet:

\begin{LTXexample}
\begin{description}
 \item[AT] das Alte Testament
 \item[NT] das Neue Testament 
\end{description} 
\end{LTXexample}


\minisec{Worterklärung, Glossar II: labeling}
Die Dokumentklassen von \KOMAScript\ stellen eine weitere Umgebung bereit, die die 
Formatierung von Schlagwortlisten noch weiter verfeinert:

\begin{lstlisting}
 \begin{labeling}[Trennzeichen]{Musterlänge}
  \item[Schlagwort] Text
 \end{labeling}
\end{lstlisting}

Das Trennzeichen steht zwischen den Schlagwörtern und der Erklärung und die Musterlänge definiert
die Breite der linken Spalte:

\begin{LTXexample}
 \begin{labeling}[=]{NSDAP}
  \item[DDP] Deutsche Demokratische Partei
  \item[DNVP] Deutschnationale Volkspartei
  \item[DVP] Deutsche Volkspartei
  \item[NSDAP] Nationalszozialistische Deutsche Arbeiterpartei
  \item[SPD] Sozialdemokratische Partei Deutschlands
 \end{labeling}
\end{LTXexample}

Wie alle Teile von \KOMAScript\ gibt es zahlreiche Möglichkeiten, die Gestaltung zu
beeiflussen. Um diese Möglichkeiten auszuschöpfen, muss man sich mit der Dokumentation 
beschäftigen.


\minisec{Geschachtelte Listen}

Ein besonderer Reiz besteht darin, dass ich die verschiedenen Listenumgebungen praktisch
beliebig komplex verschachteln lassen:

\begin{LTXexample}
\begin{itemize}
 \item Sieben freie Künste:
 \begin{itemize}
   \item Trivium:
   \begin{itemize}
    \item Grammatik
    \item Logik
    \item Rhetorik
   \end{itemize}
   \item Quadrivium:
   \begin{itemize}
    \item Arithemtik
    \item Geometrie
    \item Astronomie
    \item Musik
   \end{itemize}
 \end{itemize}
 \item Höhere Fakultäten:
 \begin{itemize}
   \item Theologie
   \item Medizin
   \item Jura
 \end{itemize}
\end{itemize}
\end{LTXexample}



\minisec{Erweiterte Möglichkeiten: das Paket \paket{enumitem}}

Nur darauf hingewiesen sei auf das Paket \paket{enumitem}, mit dem sich zahlreiche 
weitere Sonderwünsche verwirklichen lassen.
(XXX z.B. ...)
\footcite[S.~304ff.]{voss:einfuehrung}


\section{Textpassagen zitieren}

\minisec{Zitate im Fließtext}

Normale Eingabe;
Anführungszeichen am besten mit \lstinline/\enquote{}/ codieren (vgl. S. \pageref{enquote}); 
Textstellen belegen mit Fußnote und automatischer Literaturverwaltung.


\minisec{\enquote{Schlauer Spruch} am Kapitelanfang}
\dictum[xxx]{Si tacuisses, philosophus mansisses.}

\enquote{Ein häufiger anzutreffendes Element ist ein Zitat oder eine Redewendung, die rechtsbündig unter 
oder über einer Überschrift gesetzt wird. dabei werden der Spruch selbst und der Quellennachweis
in der Regel speziell formatiert.}\footcite[S. 131]{kohm:2014}

\KOMAScript\ stellt hierfür den Befehl \lstinline/\dictum[Urheber]{Spruch}/ zur Verfügung.

Die angesprochene Formatierung lässt sich in allen Details verändern.
\footcite[vgl. S. 131ff.]{kohm:2014}


\minisec{Einen Abschnitt zitieren: quote}

Die Umgebung \lstinline/quote/ eignet sich dazu, ein Zitat von der Länge eines Absatzes einzubauen:

\begin{quote}
 Seit Anfang März wusste der Diktator, dass die Tage der Diktatur gezählt waren.
\end{quote}

(Der Beginn der Kurzgeschichte \enquote{Cäsar und sein Legionär} von Bertolt Brecht.)


\minisec{Eine längere Passage zitieren: quotation}

Soll das gewünschte Zitat etwas länger ausfallen, wählt man statt dessen besser die 
Umgebung \lstinline/quotation/:

\begin{LTXexample}
\usepackage[utf8]{inputenc}
\usepackage[T1]{fontenc}

Die Kurzgeschichte \enquote{Cäsar und sein Legionär} von B. Brecht beginnt so:

\begin{quote}
 Seit Anfang März wusste der Diktator, dass die Tage der Diktatur gezählt waren.

 Ein Fremder, aus einer der Provinzen kommend, hätte die Hauptstadt vielleicht imposanter
 denn je gefunden... 
\end{quote}
\end{LTXexample}

\minisec{Ein Gedicht zitieren: verse}
\index{Gedicht} \index{Lyrik}

Das standardmäßige Umgebung \lstinline/verse/ erlaubt mit sehr wenig Mühe den Einbau von 
Gedichtpassagen, allerdings auch mit ziemlich einfacher Typografie:

\begin{LTXexample}
\minisec{Der Lorscher Bienensegen:}
\begin{verse}
 Kirst, imbi ist hucze! \\
 nu fluic du, uihi minaz, hera \\
 fridu frono in godes munt \\
 heim zi commone gisunt. \\
 sizi, sizi, bina: \\
 inbot dir sancte maria. \\
 hurulob ni habe du: \\
 zi holce ni fluc du, \\
 noh du mir nindrinnes, \\
 noh du mir nintuuinnest. \\
 sizi uilu stillo, \\
 vuirki godes uuillon.
\end{verse}
\end{LTXexample}

Strophenzwischenräume

Paket \paket{verse}? wann?

und ab wann \paket{poemscol}?


\minisec{Fremdsprachige Zitate}

Leider nur für polyglossia\footcite[S. 33 ff. ]{rouquette:2012}

\section{Fußnoten}

Fußnoten haben eine doppelte Funktion: 
Zum einen können bestimmte Informationen, die den Lesefluss des Haupttextes eher behindern,
Anmerkungen, Erklärungen, Ergänzungen oder Korrekturen in Fußnoten verlagert werden.
Zum anderen dienen -- v.a. in den Geisteswissebschaften -- Fußnoten zur Aufnahme der 
Literaturnachweise.

Diese Doppelfunktion spiegelt sich in \LaTeX\ wider, das zwei Befehle zur Fußnotenerstellung
anbietet:

\lstinline/\footnote{...}/ erzeugt eine \enquote{normale}, frei formulierte Fußnote.

\lstinline/\footcite[vordere Ergänzung][hintere Ergänzung]{bibl. Kennung}/ dient dazu,
einen Literaturverweis zu setzen. Die verwendete bibliographische Kennung muss in einer
Bibliografie-Datei aufgeschlüsselt werden und wird vom Programmsystem gemäß den Vorgaben
des eingestellten Bibliografiestils aufgelöst. Dabei lassen sich alle Rafinessen 
humanwissenschaftlicher Fußnotenkunst realisieren. 
Die Details hierzu findet man in Kapitel \ref{biblatex} auf S.~\pageref{biblatex}. 

Nach den Setzerregeln steht die Fußnotenziffer ohne Leerzeichen direkt hinter dem letzten
Satzzeichen oder Wort, auf das sich die Fußnote bezieht.
Wenn der Fußnotenaufruf -- was sehr empfehlenswert ist, um die Quelldatei gut les- und 
verarbeitbar zu halten -- die Fußnote in einer eigenen Zeile steht, sollte man durch ein 
\% -Zeichen die Umwandlung des Zeilenwechsels in ein Leerzeichen verhindern:

\begin{lstlisting}
 Hier steht ein Text%
 \footnote{lat. textus = Gewebe}
 mit Fußnoten.
\end{lstlisting}


\minisec{Fußnotenlayout mit \KOMAScript}

Verwendet man eine \KOMAScript -Klasse, so lassen sich die Fußnoten sehr leicht und vielfältig
den eigenen gestalterischen Wünschen anpassen. Dazu dienen die beiden Befehle
\lstinline/\deffootnote[Markenbreite]{Einzug}{Absatzeinzug}{Markendefinition}/
und 
\lstinline/\deffootnotemark{Markendefinition}/.

Folgende Umdefinitionen erzeugen z.B. Fußnoten mit eingeklammerten Ziffern 
in Grundschriftgröße; die jeweils erste Zeile jeder Fußnote ist eingerückt:

\begin{lstlisting}
 \deffootnote[10mm]{0mm}{0mm}{(\thefootnotemark)}
\end{lstlisting}


\minisec{Das Paket \paket{footmisc}}

Das Paket \paket{footmisc} von Robin Fairbains erlaubt zahlreiche Sonderfunktionen, die 
lediglich per Paketoption angewähl werden müssen.

Die m.E. nützlichsten Optionen sind:

\begin{description}
 \item[perpage] Die Fußnoten werden seitenweise gezählt. (2 Durchläufe nötig!)
 \item[para] Die Fußnoten werden alle zu einem Absatz zusammengefasst, das spart v.a. dann
  viel Platz und verbessert das Druckbild, wenn es sich um sehr kurze Fußnotentexte (etwa nur
  Bibelstellenangaben o.ä.) handelt.
 \item[ragged] Die Fußnoten werden im Flattersatz gesetzt.
 \item[hang] Hängender Einzug der Fußnoten (wie in diesem Buch).
  Die Breite des Einzugs wird durch \lstinline/\footnotemargin/ festgelegt.\\
  Achtung: Die Anweisung \lstinline/\setlength{\footnotemargin}{4mm}/ muss \emph{nach}
  \lstinline/\begin{document}/ stehen, der Paketaufruf aber natürlich davor!
 \item[norule] Weglassen der Fußnotenlinie.
 \item[multiple] Wenn mehrere Fußnoten direkt hintereinander stehen, wird zwischen die
  Fußnotenkennungen der Inhalt von \lstinline/\multifootsep/ eingefügt. voreingestellt ist
  ein Komma.
\end{description}



\minisec{Zweispaltige Fußnoten}

Paket ...


\minisec{Mehrere Fußnotenapparate: das Paket \paket{manyfoot}}

Das Paket \paket{manyfoot} von Alexander I. Rozhenko
erlaubt die Definition verschiedener Fußnotenapparate, um beispielsweise
Worterklärungen und Quellenhinweise voneinander zu trennen.

Auch Lesartenapparate in kritischen Editionsprojekten ließen sich mit diesem Ansatz
organisieren; allerdings bietet \paket{reledmac} hierzu die besseren Instrumente.

\section{Zeilennummern} 
\index{Zeilennummern}
\paket{lineno}
\label{zeilennummer}

Es gibt zwei grundsätzlich verschiedene Szenarien, in denen man Zeilennummern verwenden
kann. 

\begin{itemize}
 \item Einfach einen Text(abschnitt) durchnumerieren, z.B. auf einem Arbeitsblatt oder in
 einem Lehrbuch.
 \item einen Text mit Zeilennummern versehen, auf die später Bezug genommen wird, 
 z.B. im Rahmen einer großen Textausgabe. 
\end{itemize}

Hier wird nur das erste gezeigt;
Aufgabe Nr. 2 ist viel komplexer, s. Kapitel zum Erstellen einer kritischen Edition, 
S. \pageref{reledmac} 

Das Paket \paket{lineno} bietet eine Reihe von Funktionen, einen Text -- oder 
zahlreiche eingeschobene Texte wie Zitate o.ä. mit Zeilennummern als Lesehilfe zu 
versehen.


\minisec{Einen einzelnen Textabschnitt mit Zeilennummern versehen}

Wenn das Paket \paket{lineno} in der Präambel geladen worden ist, kann man im
Dokument durch den Befehl \lstinline/\linenumbers/ die Zeilennummern anschalten.
Durch \lstinline/\nolinenumbers/ wird die Nummerierung wieder abgestellt.
Auch das Ende der aktuellen Umgebung beendet die Zeilennummern.

\begin{lstlisting}
 Georg Büchners \enquote{Lenz} beginnt mit einm langen Textabschnitt:
 
 \begin{quotation}
 \modulolinenumbers[5]
 \linenumbers
 Den 20. ging Lenz durchs Gebirg. Die Gipfel und hohen Bergflächen im Schnee, die Täler
 hinunter graues Gestein...
 \end{quotation}
\end{lstlisting}

Daraus wird:
\bigskip

 Georg Büchners \enquote{Lenz} beginnt mit einm langen Textabschnitt:

 \begin{quotation}
 \modulolinenumbers[5]
 \linenumbers
 Den 20. ging Lenz durchs Gebirg. Die Gipfel und hohen Bergflächen im Schnee, die Täler
 hinunter graues Gestein, grüne Flächen, Felsen und Tannen. Es war naßkalt, das Wasser
 rieselte die Felsen hinunter und sprang über den Weg. Die Äste der Tannen hingen schwer
 herab in die feuchte Luft. Am Himmel zogen graue Wolken, aber alles so dicht, und dann 
 dampfte der Nebel herauf und strich schwer und feucht durch das Gesträuch, so träg,
 so plump. Er ging gleichgültig weiter, es lag ihm nichts am Weg, bald auf- bald
 abwärts. Müdigkeit spürte er keine, nur war es ihm manchmal unangenehm, daß er nicht
 auf dem Kopf gehen konnte.
 \end{quotation}

\minisec{Einstellmöglichkeiten von \paket{lineno}}

Das Paket bietet zahlreiche Möglichkeiten, das Erscheinungsbild der Zeilennummern zu 
verändern:

Beim Aufruf des Paketes können folgende Optionen angegeben werden:

\begin{description}
 \item[left] Die Zeilennummern stehen am linken Rand.
 \item[right] Die Zeilennummern stehen am rechten Rand.
 \item[switch] Die Zeilennummern stehen (bei zweiseitigem Layout) am äußeren Rand.
 \item[switch*] Die Zeilennummern stehen am inneren Rand.
 \item[pagewise] Die Zeilennummerierung beginnt auf jeder neuen Seite wieder bei 1.
 \item[modulo] Nur jede x-te Zeile erhält eine Nummer. 
  Das Einstellen der Schrittweite erfolgt durch \lstinline/\modulolinenumbers[...]/ 
  (sic! Die Klammern müssen eckig sein...);
  voreingestellt ist der Wert von 5.
\end{description}

Makro \lstinline/\linenumberfont/ für die Schriftart der Zeilennummer.

Länge \lstinline/\linenumbersep/ bestimmt Abstand zwischen Zeilennummer und Text.

 

\minisec{Eingeschobene Gedichte o.ä. mit Zeilennummern}

Auch die Nummerierung eines Gedichtes gelingt auf diese Weise ohne Probleme:

\begin{LTXexample}
 \begin{verse}
  \modulolinenumbers[5]
  \linenumbers
  Ist Liebe lauter nichts,\\
  wie dass sie mich entzündet.\\
  Ist sie dann gleichwohl was,\\
  wem ist ihr tun bewusst?\\
  
  Im Sommer ...\\
 \end{verse}

\end{LTXexample}



\section{Randbemerkungen (Marginalien)}
\index{Randbemerkung}
\index{Marginalien}

Mit dem Befehl \lstinline/\marginpar[links]{rechts}/ kann eine Randbemerkung gesetzt werden.
\marginpar{Randbemerkung}
Dabei ist die Bezeichnung \enquote{links} und \enquote{rechts} missverständlich: 
Randbemerkungen stehen immer außen auf den Seitenrändern; bei zweispaltigem Layout wird 
\emph{auf linken Seiten} der als \lstinline/[links]/ definierte Text ausgegeben, falls vorhanden.

Mit dem Befehl \lstinline/\reversemarginpar/ lässt sich dieses Verhalten umkehren; 
von der Benutzung ist abzuraten, da innen stehende Marginalien i.d.R. wegen der Bindung schlecht
lesbar sind.

Marginalien, die mit dem \LaTeX -Standardbefehl gesetzt sind, dürfen nicht in Gleitumgebungen
stehen und können selbst nur einfachen Text enthalten.


\minisec{Erweiterte Möglichkeiten für Marginalien}

Das paket \paket{sidenotes} von Andy Thomas bietet weitere Möglichkeiten: 
Es ist mit ihm auch Möglich, Randbemerkungen innerhalb von Gleitumgebungen zu setzen oder
Bilder
\begin{marginfigure}
 \includegraphics[width=20mm]{Caesarius_von_Heisterbach_als_Novizenmeister}
 \caption{Caesarius v. Heisterbach}
\end{marginfigure}
(inkl. Bildunterschrift) und Tabellen auf den Rand zu platzieren:

\begin{lstlisting}
 \begin{marginfigure}
 \includegraphics[width=20mm]{Caesarius_von_Heisterbach_als_Novizenmeister}
 \caption{Caesarius v. Heisterbach}
\end{marginfigure}
\end{lstlisting}


Eine weitere Option ist, die Beschriftung einer Gleitumgebung (Bild, Tabelle) auf den Rand zu
setzen, während die Gleitumgebung selbst die gesamte Breite des Textfeldes einnimmt.

Auch Bilder und Tabellen, die sich über die ganze Seitenbreite, d.h. den eigentlichen Satzspiegel
\emph{und} die Marginalspalte erstrecken, werden unterstützt.

Nähere Informationen enthält die Paketdokumentation.


\section{Texte mehrspaltig setzen}

Es gibt in \LaTeX\ zwei verschiedene Wege, zweispaltigen Satz zu erreichen:
Bereits der \LaTeX -Kern bietet die Möglichkeit, Texte zweispaltig zu setzen;
daneben gibt es das Paket \paket{multicol}, dass mehrspaltige Einschübe in einem ansonsten
einspaltigen Dokument ermöglicht.

Beide Verfahren haben ihre Berechtigung, denn es ergeben sich jeweils verschiedene
Vor- und Nachteile bzw. Einschränkungen:

\minisec{Das ganze Dokument zweispaltig setzen}

Wird als Dokumentklassen-Option \lstinline/twocolumn/ angegeben, wird das Dokument 
zweispaltig gesetzt. 
Der Titel mit Datum, Autorangabe, Abstract etc. werden als einspaltiger Kopf über beide 
Spalten gesetzt.


\minisec{Zweispaltige Einschübe in einem einspaltigen Dokument I: \LaTeX}

Daneben bietet \LaTeX die Möglichkeit, in einem Dokument eine zweispaltige Passage
zu integrieren. Die Passage wird mit dem Befehl \lstinline/\twocolumn[xxx] /
eingeleitet. Die angegebene Überschrift wird einspaltig über beide Spalten gesetzt.
Außer einer echten Überschrift (z.B. \lstinline/\section{}/ etc. sind hier auch
längere Einleitungstexte etc. möglich.

Mit dem Befehl \lstinline/\onecolumn/ wird wieder auf einspaltigen Satz zurückgestellt.

Beide Kommandos beginnen jeweils eine neue Seite, was ihre praktische Benutzbarkeit m.E.
stark einschränkt.


\minisec{Mehrspaltige Einschübe in einem einspaltigen Dokument II: \paket{multicol}}

\begin{multicols}{2}
 Die umfassendste Funktionalität bietet das Paket \paket{multicol} von Frank Mittelbach.
 Mit ihm lassen sich beliebige Textpassagen in mehreren Spalten in ein ansonsten
 einspaltiges Dokument einfügen.
 
 Der Beginn einer mehrspaltigen Passage wird durch \lstinline/\begin{multicols}{Zahl}/ 
 eingeleitet; die entsprechende Anweisung \lstinline/\end{multicols}/ schaltet wieder
 auf einspaltigen Satz um. \lstinline/Zahl/ steht dabei für die Anzahl der gewünschten 
 Spalten.                                            
\end{multicols}

\setlength{\columnseprule}{0.4pt}
\begin{multicols}{3}
 Dabei liegt es in der Veranwortung des Nutzers, nicht zuviele Spalten zu verlangen,
 da sonst die einzelnen Spalten zu schmal werden und sich Aussehen und Lesbarkeit erheblich
 verschlechtern.
 
 Faustregel: Der Wortabstand soll niemals größer sein als der Zeilenabstand.
 Evtl. kann es helfen, mehrspaltige Abschnitte im Flattersatz zu setzen.
 In typographischer Hinsicht ist dies ein sehr schlechter Absatz...
 
 Um dem Leser zu ermöglichen, die Spalten besser wahrzunehmen kann der Längenwert 
 \lstinline/columnseprule/ auf einen Wert größer 0 gestellt werden;
 in diesem Beispiel liegt er bei 0.4pt: \lstinline/\setlength{\columnseprule}{0.4pt}/.
 
 Daneben bietet das Paket weitere Einstellmöglichkeiten, wie. z.B. den Spaltenabstand
 oder die Farbe der Trennlinie.
\end{multicols}



\section{Das Konzept der ``Gleitumgebung''}
\dictum[Heraklit?]{Alles fließt.}

\minisec{Gleitumgebungen beschriften: Das Paket \paket{caption}}

\section{Grafiken einbinden}

Voraussetzung: Paket \paket{graphicx} in Dokumentspräambel einbinden:
\lstinline/\usepackage{graphicx}/.

Dann fügt der Befehl \lstinline/\includegraphics[Optionen]{Name}/ an der jeweiligen Stelle
die angegebene Grafik (als eigenen Absatz) ein.
Als Option sollte z.B. die gewünschte Abbildungsgröße eingestellt werden.
Wieder sind sowohl absolute Angaben möglich als auch Bezugnahmen z.B. auf 
\lstinline/\textwidth/.

Meist möchte man die Abbibldung jedoch in eine Gleitumgebung einfügen, so dass das Satzprogramm
flexibler die Positionierung entscheiden kann (s. vorheriger Abschnitt).
Dazu dient die Umgebung \lstinline/figure/.
Dann ist auch die Angabe einer Bildunterschrift mit dem Paket \paket{caption} möglich.

Ggf. empfiehlt sich auch noch das Eibetten in eine \lstinline/center/-Umgebung,
um das Bild auf die Mitte des Satzspiegels zu zentrieren.

Alles zusammen sieht dann so aus:

\begin{lstlisting}
\begin{figure}
 \begin{center}
  \includegraphics[width=.5\textwidth]{Caesarius_von_Heisterbach_als_Novizenmeister}
  \caption{Caesarius von Heisterbach als Novizenmeister}
 \end{center}
\end{figure} 
\end{lstlisting}


\begin{figure}
 \begin{center}
  \includegraphics[width=.7\textwidth]{Caesarius_von_Heisterbach_als_Novizenmeister}
  \caption{Caesarius von Heisterbach als Novizenmeister}
 \end{center}
\end{figure}


\section{Diagramme zeichnen}
allgemeine vs. spezielle Lösungen

\subsection{Der unsaubere Weg: Open Office und co. benutzen}

Einbinden der fertigen Abb. als Grafik


\subsection{Die beiden wichtigsten Pakete: pstricks und tikz}

\minisec{pstricks}
\paket{pstricks}

\minisec{tikz}
\paket{tikz}

\subsection{Linguistische Strukturen}
\index{Linguistik} \index{x-bar-Schema} \index{Phrasenstruktur}
\paket{covington}
\footcite{roemer:dtk2008}
\footcite{roemer:dtk2016}


\minisec{Baumdiagramme}
\index{Baumdiagramm} \index{Stemma}

Das Paket \paket{forest} von Saso Zivanovic erlaubt einfache und sehr komplexe Baumsstrukturen:

\begin{LTXexample}
 \begin{forest}
  [VP
    [DP]
    [V'
      [V]
      [DP]
    ]
  ]
\end{forest}
\end{LTXexample}

\subsection{Ausgewählte Anwendungen}


\minisec{Zeitschienen}


\minisec{Stammbäume}


\minisec{Statistiken visualisieren}

Torten- und Balkendiagramme... 

\paket{datatool} ?



\section{Tabellen erstellen}

Komplexes Thema, Spezialliteratur gerechtfertigt.\footcite{voss:tabellen}


\paket{tabularx}

\paket{longtable}

\minisec{Tabellen drehen}
\index{Drehen von Texten}

\paket{rotatebox}


\section{Umgang mit Bibelstellen}
\index{Bibelstellen}

\paket{bibleref}

\paket{bibleref-german}

\paket{bibleref-parse}

Zur Registererstellung vgl. Abschnitt \ref{Bibelstellenregister} auf Seite \pageref{Bibelstellenregister}



\section{Lyrik-Satz}
\index{Gedicht} \index{Lyrik}

\paket{poemscol} (?) und andere?

\section{Dramen}
\paket{covington}
\paket{dramatist}
\footcite[xxx]{lesetypografie}


\section{Querverweise im Text}

\lstinline/\label{key}/

\lstinline/\ref{key}/

\lstinline/\pageref{key}/


\section{Eigene Kommandos und Umgebungen definieren}
\label{makros}

\minisec{Kommandos}

\minisec{Umgebungen}




\chapter{Textpassagen in nicht-lateinischen Alphabeten einbetten}

\section{Allgemeines: babel, Unicode}

Grunndlegend ist der Artikel von Axel Kielhorn in der DTK.%
\footcite{kielhorn:dtk2014}

\section{Eingabe von Unicode-Sonderzeichen}
\label{unicodeeingabe}

\minisec{Eingabe direkt mit der Tastatur}

Lohnend besonders bei den modernen Fremdsprachen: Anschaffen einer Tastatur in der jeweiligen
Sprache.

Problem im altphilologischen Bereich: Akzente und (bei Hebräisch) Vokalzeichen sind heute
nicht mehr verwendet und fehlen deshalb auf den heutigen Tastaturen.


\minisec{Auswahl über Maus-gestützte tools}
KCharselect

\begin{figure}
 \includegraphics[width=\textwidth]{kcharselect}
 \caption{Mit dem KDE-Programm KCharselect kann man Unicode-Buchstaben immerhin mit der Maus
 auswählen.}
\end{figure}

\minisec{Eingabe über die Zeichennummer}

Vgl. Abschnitt \ref{utf8codes} auf Seite \pageref{utf8codes}.

\section{Griechisch}
\index{Griechisch}

Es gibt (mindestens...) drei verschiedene Möglichkeiten, griechische Textpassagen einzubinden;
je nach Aufgabenbreich sind diese verschieden gut geeignet:


\minisec{Möglichkeit I: Einzelbuchstaben}

Die erste Möglichkeit besteht in der Eingabe m.H. der Einzelnbuchstaben-Symbole des Paketes
\paket{textgreek}; das Verfahren ist in Abscnitt \ref{griechEinzelbuchstaben}, 
S.~\pageref{griechEinzelbuchstaben}, beschrieben.

Dieses Verfahren eignet sich eigentlich nur für ganz kurze Einsprengsel im Umfang von 
einzelnen Zeichen bis maximal ca. einem Wort.


\minisec{Möglichkeit II: Transkription mit \paket{ibycus}}

Die zweite Methode eignet sich demgegenüber hervorragend, wenn es darum geht auch etwas 
längere (alt-)griechische Passagen in ein ansonsten deutschsprachiges Dokument -- m.H. eines
deutschen Computers mit QWERTZ-Tastatur -- einzugeben.

Dabei wird für die griechischen Buchstaben eine Art Transkription in lateinischen Buchstaben
vorgenommen, die sich bei etwas Übung sehr leicht benutzen lässt.

Das Paket \paket{ibycus}%
\footnote{Paketdokumentation: texdoc ibycus-babel}
definiert dabei eine Art Pseudosprache für \paket{babel},
in der Dokumenpräambel ist also anzugeben:

\begin{lstlisting}
 \usepackage[ibycus,ngerman]{babel}
\end{lstlisting}

Im Dokument stehen dann eine Umgebung \lstinline/ibycus/ sowie ein Befehl \lstinline/\ibygr{}/
zur Verfügung:


Das wird zu:

 \begin{ibycus}
  (Hrodo'tou Qouri’ou i(stori’hs a)po’decis h(’de,
  ...
  h(‘n ai)ti’hn e)pole’mhsan a)llh’loisi. 
  \end{ibycus}

  Und jetzt ein einzelnes griechisches Wort --  \ibygr{a)rxai=a gra’mmata}  -- im Text.





\minisec{Möglichkeit III: Direkteingabe von Unicode-Text}

Die dritte Möglichkeit besteht darin, die griechischen Schriftzeichen direkt als Unicode-Zeichen
in das \LaTeX -Dokument einzufügen. Dazu reicht es aus, dem Paket \paket{babel} die 
(alt-)griechische Sprache als zusätzliche Option anzugeben: 
\lstinline/\usepackage[polutonikogreek,ngerman]{babel}/.
\footnote{Achtung! Bei den meisten \LaTeX -Installationen im Rahmen von Linux-Distributionen muss
man Pakete nachinstallieren!}
Damit \LaTeX\ den griechischen Font richtig ansprechen kann, braucht auch das Paket \paket{fontenc} 
eine Modifikation: \lstinline/\usepackage[OT1,T1]{fontenc}/.

Dann kann mit \lstinline/\selectlanguage{polutonikogreek}/ auf Griechisch umgeschaltet werden.

Der folgende Bibelvers wurde aus dem Internet unverändert in das \LaTeX -Dokument
kopiert. Derzeit werden nicht alle Zeichen in meinem KDE-Editor (kile) korrekt dargestellt (vgl. Abb.);
dennoch gibt \LaTeX\ alle Zeichen richtig wider:

\begin{figure}
 \includegraphics[width=\textwidth]{ersatzdarstellung}
 \caption{KDE kann nicht alle aus dem Internet kopierten Unicode-Zeichen korrekt darstellen;
 Buchstaben mit Akzent und Spiritus bekommen eine (nichtssagende) Ersatzdarstellung!}
\end{figure}


\begin{otherlanguage}{polutonikogreek}
Οὕτως γὰρ ἠγάπησεν ὁ Θεὸς τὸν κόσμον, ὥστε τὸν Υἱὸν τὸν μονογενῆ ἔδωκεν, 
ἵνα πᾶς ὁ πιστεύων εἰς αὐτὸν μὴ ἀπόληται ἀλλ’ ἔχῃ ζωὴν αἰώνιον.
\end{otherlanguage}
(Joh. 3,16)

Diese Methode dürfte sich in der Praxis am ehesten eignen, wenn bereits ein Textcorpus
(z.B. via Internet) ediert und Unicode-Erfasst vorliegt, und Textpartien ins eigene Dokument
ohne Veränderungen übernommen werden sollen -- so wie z.B. bei Zitaten aus der Bibel oder
der klassischen Literatur.

Will man selbst (alt-)griechische Textpassagen schreiben, scheint das Ibykus-Verfahren
wesentlich angenehmer und effizienter.


\section{Hebräisch}
\index{Hebräisch}

Das Paket \paket{cjhebrew} von Christian Justen eignet sich hervorragend für 
Theologen und Orientalisten, denn es erlaubt auch die Vokalisierung sowie die Verwendung 
von Akzenten. Auch das Problem der rechts- bzw. linksläufigen Schriften wird sehr
Anwenderfreundlich -- im Sinne von Anwendern an einem westlichen PC mit ansonsten 
lateinischen Alphabet -- gelöst:

Nach \lstinline/\usepackage{cjhebrew}/%
\footnote{Eine besondere Angabe z.B. bei \paket{babel} ist nicht nötig.}
steht v.a. der Befehl \lstinline/\cjRL{}/ 
-- für hebräische Einzelworte im laufenden Absatz -- sowie die 
Umgebung \lstinline/cjhebrew/ 
-- für ganze Absätze in Hebräisch -- zur Verfügung.

Beide drehen die Folge der Schriftzeichen automatisch um, so dass hebräische Textpartien
ganz gewohnt eingegeben werden können. Dabei sind für die einzelnen Zeichen sehr leicht 
merkbare Codes definiert worden:

Der Beginn des biblischen Buches Genesis -- \enquote{Im Anfang schuf Gott Himmel und Erde...} --
wird so eingegeben:

\begin{lstlisting}
\begin{cjhebrew}
 b*e:re'+siyt b*ArA' 'E:lohim 'et ha+s*amayim w:'et hA'ArE.s;
\end{cjhebrew} 
\end{lstlisting}

Mit etwas Eingewöhnung arbeitet man mit diesem System \emph{wesentlich} schneller und 
angenehmer, als wenn man etwa die Unicode-Zeichen mit Hilfe eines grafischen Auswahlwerkzeuges
einzeln auswählt...

Die Bearbeitung mit \LaTeX\ ergibt:

\begin{cjhebrew}
 b*e:re'+siyt b*ArA' 'E:lohim 'et ha+s*amayim w:'et hA'ArE.s;
\end{cjhebrew}

Wegen der Einzelheiten der Zeichen-, Vokal-, Akzent- und Satzzeichen-Codierung (inkl. z.B. 
Mem finalis) sei auf die Paketdokumentation verwiesen.

\section{Russisch}
\index{Kyrillisch}
\index{Russisch}

\section{Koptisch}
\index{Koptisch}

\section{Altkirchenslawisch}
\index{Altkirchenslawisch}

\paket{churchslavonic}


\section{Arabisch}
\index{Arabisch}

Ist \paket{arabtex} noch aktuell?

\section{Hieroglyphen}
\index{Hieroglyphen}
\index{Ägyptologie}

\paket{hieroglf} --  \enquote{The Poor Man’s Hieroglyphic Font}


 \pmglyph{K:l-i-o-p-a-d:r-a} 
 

\section{Keilschrift}
\index{Keilschrift}

\paket{archaic} und andere...

\section{Runen}
\index{Runen}

Das Paket \paket{runic} von Peter Wilson stellt die Runen des sog. Futhark-Alphabets bereit.
Nach \lstinline/\usepackage{runic}/ gibt es den Befehl \lstinline/\textfut{}/, der 
die Runen ausgibt. Dies ist die Transkription:

\begin{center}
\begin{tabular}{ll}
 \lstinline/F/ &	\textfut{F} \\
 \lstinline/U/ &	\textfut{U} \\
 \lstinline/\Fthorn/ &	\textfut{\Fthorn} \\
 \lstinline/A/ &	\textfut{A} \\
 \lstinline/R/ &	\textfut{R} \\
 \lstinline/K/ &	\textfut{K} \\
 \lstinline/G/ &	\textfut{G} \\
 \lstinline/W/ &	\textfut{W} \\
 \lstinline/H/ &	\textfut{H} \\
 \lstinline/N/ &	\textfut{N} \\
 \lstinline/I/ &	\textfut{I} \\
 \lstinline/J/ &	\textfut{J} \\
 \lstinline/Y/ &	\textfut{Y} \\
 \lstinline/P/ &	\textfut{P} \\
 \lstinline/X/ &	\textfut{X} \\
 \lstinline/S/ &	\textfut{S} \\
 \lstinline/T/ &	\textfut{T} \\
 \lstinline/B/ &	\textfut{B} \\
 \lstinline/E/ &	\textfut{E} \\
 \lstinline/M/ &	\textfut{M} \\
 \lstinline/L/ &	\textfut{L} \\
 \lstinline/\Fng/ &	\textfut{\Fng} \\
 \lstinline/D/ &	\textfut{D} \\
 \lstinline/O/ &	\textfut{O} \\
 \lstinline/:/ &	\textfut{:} \\
 \end{tabular}
 \end{center}


\section{Phonetische Alphabete}


\section{Kurzschriften}
\index{Kurzschrift}
\index{Schnellschrift}

\minisec{Tironische Noten}
\index{Tironische Noten}


\minisec{Deutsche Einheits-Kurzschrift}
DEK\footcite{sarman:dtk2009/1}
\index{Deutsche Einheits-Kurzschrift}
\index{DEK}


\minisec{Pitman-Kurzschrift}
\index{Pitman}


\chapter{Texte parallel setzen}

\section{Wortweise: Interlinearglossen}
\index{Interlinearübersetzung}

Das Paket \paket{covington} von Michael Covington stellt einige primär für Linguisten nützliche
Befehle zur Verfügung. Mit seiner Hilfe ist es relativ einfach, eine einfache 
Interlinearglosse zu erstellen, d.h. Wort für Wort zu übersetzen oder zu kommentieren:

\begin{lstlisting}
\gll La  neige,  qui {n'a pas} cesse       de tomber 
     Der Schnee, der {nicht hat} aufgehört zu fallen 
\glt Der Schnee, der nicht zu fallen aufgehört hat
\glend 
\end{lstlisting}
stand oder
\gll La  neige,  qui {n'a pas} cesse       de tomber 
     Der Schnee, der {nicht hat} aufgehört zu fallen 
\glt Der Schnee, der nicht zu fallen aufgehört hat
\glend

Zu beachten ist, dass es neben dem Kommando \lstinline/\gll/ auch ein Kommando 
\lstinline/\glll/ gibt, das die Zusammenstellung von drei Zeilen ermöglicht.

\textbf{Achtung!} Die neueste Version von \paket{covington} stammt aus dem Jahr 
2001 und verwendet altermtümliche Fontbefehle. Damit das paket mit modernen Dokumentklassen 
wie \KOMAScript\ zusammenarbeitet, war es nötig, im Paket die Zeilem 225-227 auszukommentieren.

\section{Spaltenweise}

Das Paket \paket{parallel} von Matthias Eckermann stellt eine Umgebung 
\lstinline/Parallel/ zur Verfügung, in der mit den Befehlen 
\lstinline/\ParallelLText{...}/ und 
\lstinline/\ParallelRText{...}/ die jeweils links- bzw. rechtsstehenden Texte
angegeben werden können.

Durch den Befehl \lstinline/\ParallelPar/ wird ein neuer \enquote{Doppelabsatz} begonnen.

\begin{lstlisting}
 \linenumbers
 \begin{Parallel}{.45\textwidth}{.45\textwidth}
  \ParallelLText{Gallia est omnis divisa in partes tres ...}
  \ParallelRText{Gallien als ganzes ist in drei Teile...}
  \ParallelPar
  \ParallelLText{Hi omnes...}
  \ParallelRText{Sie alle...}
 \end{Parallel}
\end{lstlisting}

Beim Erzeugen der \lstinline/Parallel/-Umgebung ist für die linke und rechte Seite ihre
jeweilige Breite anzugeben; das kann als absolute Angabe (in mm oder cm) oder als 
Bezugnahme zu einem Wert wie \lstinline/\textwidth/ (der Breite des aktuellen 
Satzspiegels) erfolgen.

Das obige Beispiel erzeugt folgende Ausgabe:
\bigskip 

\linenumbers
\begin{Parallel}{.45\textwidth}{.45\textwidth}
  \ParallelLText{Gallia est omnis divisa in partes tres, 
    quarum unam incolunt Belgae,
    aliam Aquitani,
    tertiam, qui ipsorum lingua Celtae, nostra Galli appellantur.}
  \ParallelRText{Gallien als ganzes ist in drei Teile gegliedert,
    deren erster die Belger bewohnen,
    den zweiten die Aquitanier,
    den dritten jene, die in ihrer eigenen Sprache Kelten, in der unseren Gallier genannt werden.}
  \ParallelPar
  \ParallelLText{Hi omnes lingua institutis legibus inter se differunt.}
  \ParallelRText{Sie alle unterscheiden sich hinsichtlich der Sprache, 
    der (staatlichen) Einrichtungen und der Gesetze von einander.}
 \end{Parallel}
\nolinenumbers

(Zu der Zeilennummerierung vgl. Abschnitt \ref{zeilennummer} auf S. \pageref{zeilennummer}.)

\minisec{Das Paket \paket{paracol}}

\paket{paracol}

\section{Seitenweise}

\paket{ledpar}\footcite[S. 175ff.]{rouquette:2012}


\chapter{Kritische Apparate setzen}
\dictum[?]{Kritik ist überall, zumal in Deutschland, nötig.}

\index{Apparat} \index{Varianten}
\label{reledmac}

\enquote{Edieren ist eine Erziehung zur Bescheidenheit [...] Es ist ferner eine Erziehung zur 
Genauigkeit, wie alle Philologie.}%
\footnote{Erich Trunz, Ein Tag aus Goethes Leben. Acht Studien zu Leben und Werk, München 1999, S.~213.}

Bezug zu TUSTEP?
Recherche zu anderen Alternativen?

\paket{reledmac}, \paket{ledpar} und \paket{ednotes} (wird nicht erklärt.)
\footcite[S. 165ff.]{rouquette:2012}


\chapter{Literatur und Zitate automatisch verwalten}
\dictum[U. Eco, Name der Rose]{...}
\label{biblatex}


\section{Der (neue) Standard: biblatex und biber}

\paket{biblatex}
Vgl.\footcite[S. 79ff.]{rouquette:2012}
\footcite{voss:bibliografien}
\footcite{wassenhoven:dtk2008/2}
\footcite{wassenhoven:dtk2008/4}



\section{Aufbau der Bibliografie-Datenbank}

\subsection{Grundlegender Aufbau}

\subsection{Schlüsselvergabe}

\subsection{Publikationstypen und ihre Datenfelder}

\subsubsection{Bücher}

\subsubsection{Zeitschriften}

\subsubsection{Artikel}

\subsubsection{Internet-Resourcen}

\subsubsection{Übersicht: Publikationstypen und Datenfelder}

alle auflisten....


\section{Zitate}

\lstinline/\cite{key}/

\lstinline/\footcite{key}/

\lstinline/\nocite{key}/




\section{Bibliographiestile}

\minisec{Standardstile von biblatex}


\minisec{Nützliche Bibliografiestile, die nachinstalliert werden müssen}

Installation als Paket nötig:
...



\section{Ein Beispiel für Historiker: Quellen und Sekundärliteratur}
\index{Quellenverzeichnis}
\index{Bibliographie}
\index{Literaturverzeichnis}
\index{Sekundärliteratur}


\chapter{Texte durch Register erschließen}
\dictum[Schiller, Räuber]{Dein Register hat ein Loch.}
\index{Register} \index{Index}

\section{Allgemeines}

Sortierproblem bei Umlauten


\section{Mehrere Register zu einem Dokument}

\minisec{Problem und grundsätzliche Strategie}
\cite{voss:einfuehrung}

\minisec{Lösungsansatz I: imakeidx}
\paket{imakeidx} 

\minisec{Lösungsansatz II: splitidx}
\paket{splitidx}

\section{Bibelstellenregister}

\label{Bibelstellenregister}
\index{Bibelstellenregister}
\paket{bibleref-german}

\section{Worthäufigkeit}
\index{Häufigkeit (von Wörtern)}

\section{Reimwörter}
\index{Reimwörterbuch}


\chapter{Prüfungen aufsetzen: exam}

exam\footcite{ziegenhagen:dtk2016/2}


\chapter{Präsentationen gestalten: beamer}

beamer\footcite{voss:praesentationen}


\chapter{Eigene \LaTeX -Erfindungen dokumentieren}

Man muss kein besonders gewiefter \TeX niker sein, um in die Verlegenheit zu kommen, eigene 
\LaTeX -``Erfindungen`` dokumentieren zu müssen
-- und sei es nur ein einzelnes kleines \lstinline/\renewcommand/
im Rahmen eines Projektes mit mehreren Mitarbeitern.

Auf den zweiten Blick ist dies für Geisteswissenschaftler gar nicht so anders, 
als die anderen Dinge auch, die Philologen mit Textverarbeitungsprogrammen so anstellen:

dokumentierter text  ---  dokumentierender text ...


\minisec{Wiedergabe der \TeX -typischen Logos}

Das Paket \paket{hologo} von Heiko Oberdiek stellt den Befehl \lstinline/\hologo{Name}/ zur 
Verfügung, das u.a. folgende Logos erzeugen kann:

\begin{center}
 \begin{tabular}{ll}
  (La)TeX & \hologo{(La)TeX} \\
  AmSLaTeX & \hologo{AmSLaTeX} \\
  AmSTeX & \hologo{AmSTeX} \\
  biber & \hologo{biber} \\
  BibTeX & \hologo{BibTeX} \\
  HanTheThanh & \hologo{HanTheThanh} \\
  KOMAScript & \hologo{KOMAScript} \\
  La & \hologo{La} \\
  LaTeX & \hologo{LaTeX} \\
  LaTeX2e & \hologo{LaTeX2e} \\
  LaTeX3 & \hologo{LaTeX3} \\
  LaTeXe & \hologo{LaTeXe} \\
  LuaLaTeX & \hologo{LuaLaTeX} \\
  LuaTeX & \hologo{LuaTeX} \\
  LyX & \hologo{LyX} \\
  METAFONT & \hologo{METAFONT} \\
  MetaFun & \hologo{MetaFun} \\
  METAPOST & \hologo{METAPOST} \\
  MetaPost & \hologo{MetaPost} \\
  MiKTeX & \hologo{MiKTeX} \\
  teTeX & \hologo{teTeX} \\
  TeX & \hologo{TeX} \\
  Xe & \hologo{Xe} \\
  XeLaTeX & \hologo{XeLaTeX} \\
  XeTeX & \hologo{XeTeX} \\
 \end{tabular}

\end{center}





\minisec{Listings einbinden}

\paket{listings}


\minisec{Latex-Quelltext und seine Ausgabe wiedergeben}

\paket{showexpl}


\chapter{Anhang}

\section{Ein Beispiel, das (fast) alles kann}

In der folgenden Musterdatei wird (fast) alles vorgeführt, was das Skript erklärt.
Sie kompiliert in kile durch ALT+6...

\lstinputlisting{lfgw-musterdatei}


\section{Unicode}
\label{unicode} \index{Unicode}

\subsection{Einstellen des Editors auf Unicode}

\subsection{Umcodieren vorhandener Dateien}

Programm recode


\subsection{Häufig benötigte Unicode-Zeichen}

\label{utf8codes}

\section{Wie installiere ich die Software/Pakete etc.}

\minisec{Die radikale Alternative: Cloud-Lösung}
\index{overleaf}
\index{sharelatex}


\minisec{Windows: miktex}
\index{miktex}

\minisec{Standardwerkzeug der Linux-Distribution}

\minisec{ctan}
\index{ctan}


\section{Woher beziehe ich Dokumentation zu den Paketen?}

\minisec{Welche Pakete könnten interessant sein?}

ctan

\minisec{Wie funktionieren die schon installierten Pakete?}

texdoc PAKETNAME


\section{Welche Bücher sollte ich mir kaufen?}

Erster Schritt: Dokument \enquote{\LaTeXe -Kurzbeschreibung} mit 
\lstinline/texdoc lshort/. Dieses Dokument (ca. 50 Seiten) sollte man am besten ausdrucken.

Einen grundlegenden Überblick über das Gesamtsystem bietet
\cite{voss:einfuehrung}

Wenn man sich zur Benutzung der \KOMAScript -Klassen entscheidet, ist die ultimative Referenz,
mit der man erst das ganze Paket ausnutzen kann:
\cite{kohm:2014}

Die einzige Spezialmonographie zum Thema \LaTeX\ in den Geisteswissenschaften ist
\cite{rouquette:2012}

Ideal zum Nachschlagen bestimmter Befehle eignet sich:
\cite{voss:referenz}

Spezialtitel, je nach individuellem Bedürfnis:

\cite{voss:praesentationen}

\cite{voss:bibliografien}

\cite{voss:pstricks}


\section{Bücher veralten. Wer hält mich auf dem Laufenden?}

dante e.V.


\printbibliography



\chapter{Register}
\printindex          % der "allgemeine" Index
\printindex[pakete]


\vfill
\minisec{Kolophon}

opensuse 13.2 auf lenovo ideapad s10e von 2008

kile

pdflatex mit \KOMAScript 

Klasse scrbook

geometry, DIV,  BCOR ...

babel

lstlisting

biblatex und biber

Bibliografiestil der Historischen Zeitschrift (\lstinline/\usepackage[style=historische-zeitschrift]{biblatex}/)

imakeidx
\end{document}
