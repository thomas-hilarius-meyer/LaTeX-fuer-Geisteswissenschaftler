% !TeX root = lfgw.tex
% !TEX TS-program = pdflatex
% !BIB TS-program = biber
\RequirePackage[ngerman=ngerman-x-latest]{hyphsubst}
\documentclass[%
   %%%draft=false,%
   fontsize=11pt,%
   paper=17cm:24cm,%
   DIV=13,%
   pagesize%
   ]{scrbook}

\usepackage[utf8]{inputenc}
\usepackage[OT1,T1]{fontenc}
\usepackage[ibycus,polutonikogreek,english,french,ngerman]{babel}
\usepackage{libertine}
\RequirePackage{libertinegc}
\usepackage[supstfm=libertinesups,%
   supscaled=1.2,%
   raised=-.13em]% match XHeight of libertine
   {superiors}
\usepackage[scaled=0.85]{beramono}
\usepackage{textcomp}
%\usepackage{euler}
\usepackage{yfonts}
\usepackage{textgreek}
\usepackage[%
   final,%
   tracking=smallcaps,%
   %%%expansion=alltext,%
   protrusion=true%
   ]{microtype}%
\SetTracking{encoding=*,shape=sc}{50}%
\UseMicrotypeSet[protrusion]{basicmath} % disable protrusion for tt fonts

\usepackage[autostyle]{csquotes}
\usepackage[newcommands]{ragged2e}

\usepackage{graphicx}
\usepackage{grffile}
\usepackage[a4,center,cross]{crop}

\usepackage{listings}
\lstset{inputencoding=utf8,
  language=[LaTeX]{TeX},
  numbers=left, 
  %stepnumber=3, 
  numberfirstline=false,  
  numberstyle=\tiny\textsf,
  basicstyle=\footnotesize\ttfamily,
  %frame=tlrb,
  breaklines=true,
  postbreak=\mbox{$\hookrightarrow$},
  %showstringspace=false, 
  captionpos=b,
  literate={ä}{{\"a}}1{ö}{{\"o}}1{ü}{{\"u}}1{ß}{{\ss}}1
  }

\usepackage{showexpl}

\usepackage[pagewise]{lineno}

\usepackage{sidenotes}

\usepackage{multicol}

\usepackage{forest}

\usepackage{covington}
%%%\renewcommand{\it}{}
%%%\renewcommand{\rm}{}

\usepackage[hang]{footmisc} %%%MS: Evtl. besser direkt \deffootnote

\usepackage{parallel}

\usepackage{cjhebrew}

\usepackage{runic}

\usepackage{hieroglf}

\usepackage[hyphens]{url}

\usepackage{hologo}

\usepackage{imakeidx}
\indexsetup{level = \subsection*, toclevel = subsection, noclearpage, headers = {\indexname}{\indexname}}

\makeindex[                title = {Allgemeiner Index}]
\makeindex[name = pakete,  title = {Verzeichnis der Paketnamen}]

\usepackage[style=verbose-inote,pageref=true,backend=biber]{biblatex}
%\usepackage[style=historische-zeitschrift]{biblatex}
\addbibresource{lfgw-bibliographie.bib}
\defbibheading{bibliography}{\chapter{#1}} %%%Besser bei \addchap bleiben

\newcommand{\paket}[1]{\textsf{#1}\index[pakete]{#1}}
\newcommand{\Package}[1]{\paket{#1}}%zwecks kompatibilität mit DTK

\newcommand{\LuaLaTeX}{\hologo{LuaLaTeX}}
\newcommand{\METAFONT}{\hologo{METAFONT}}
\newcommand{\pdfLaTeX}{\hologo{pdfLaTeX}}
\newcommand{\XeLaTeX}{\hologo{XeLaTeX}}

\setlength{\footnotemargin}{4mm}

\tolerance 1414
\hbadness 1414
\emergencystretch 1.5em
\hfuzz 0.3pt
\widowpenalty=10000
\displaywidowpenalty=10000
\clubpenalty=5000
\interfootnotelinepenalty=9999
\brokenpenalty=2000
\vfuzz \hfuzz
%%%\raggedbottom
\endinput