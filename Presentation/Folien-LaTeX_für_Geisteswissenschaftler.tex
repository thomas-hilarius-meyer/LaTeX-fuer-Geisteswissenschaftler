\documentclass[12pt,ngerman]{beamer}
\usepackage[utf8]{inputenc}
\usepackage[T1]{fontenc}
\usepackage{booktabs}
\usepackage{babel}
\usepackage{graphicx}
\usepackage{csquotes}
\usepackage{xcolor}


% --- Sorry, aber das geht bei mir im Moment nicht? Vllt. Tex-Installation zu alt? --------- thm 2018-08-17
%\usepackage[sfdefault]{plex-sans}
\usetheme[progressbar=frametitle]{metropolis}           % Use metropolis theme

\title{\LaTeX{} für Geisteswissenschaftler -- Eine Einführung}
\date{15. September 2018}
\author{Thomas Hilarius Meyer, Martin Sievers und Uwe Ziegenhagen}

\makeatletter
%\setlength{\metropolis@titleseparator@linewidth}{1pt}
%\setlength{\metropolis@progressonsectionpage@linewidth}{1pt}
%\setlength{\metropolis@progressinheadfoot@linewidth}{1pt}
\makeatother

\begin{document}

\begin{frame}
	 \maketitle
\end{frame}

\begin{frame}
\frametitle{Einleitung}

\begin{itemize}
\item \LaTeX-Literatur für MINT-Fächer ist reichlich vorhanden
\item Geisteswissenschaften etwas \enquote{zu kurz} gekommen
\item Zusammenschluss verschiedener Autoren
\item Ziel: \LaTeX-Einführung für Geisteswissenschaftler, mittelfristig als Buch
\item Zeithorizont: \ldots
\end{itemize}

\end{frame}

\begin{frame}
\frametitle{Autoren}

Wer ist beteiligt?

\begin{itemize}
\item Lukas C. Bossert
\item Jürgen Fenn
\item Axel Kielhorn
\item Thomas H. Meyer
\item Craig Parker-Feldmann
\item Dr. Philipp Pilhofer
\item Christine Römer
\item Martin Sievers
\item Dr. Uwe Ziegenhagen, Köln
\end{itemize}
\end{frame}

\begin{frame}
\frametitle{Wie es begann\ldots}

\begin{itemize}
\item Sommer 2016: Diskussion auf dante-ev
\item Erstellen eines ersten Skripts
\item ab November 2016: Gründung der Arbeitsgruppe
\item Resultat:
  \begin{itemize}
  \item Mehr thematische Vielfalt
  \item Mehr Expertise
  \item größere technische Vielfalt (\TeX-Installationen, Versionen, Editoren\ldots)
  \item größerer Abstimmungsbedarf
  \item (vgl. Gesetz von Brooks\ldots)
  \end{itemize}
\end{itemize}
\end{frame}


\begin{frame}
  \frametitle{Das Vorbild}

  Maieul Rouquette (Dante-Ehrenpreisträger 2017)

  Le seul livre sur \LaTeX{} sans une seule equation!
\end{frame}


\begin{frame}
\frametitle{Themenüberblick}

\begin{itemize}
\item \LaTeX-basics \ldots
  \item \ldots mit Schwerpunkten: Sonderzeichen, spezielle Schriftarten, Lyrik- und Dramensatz
\item Diagramme
\item Nicht-lateinische Alphabete
\item Apparatsatz
\item Texte parallelisieren
\item Literaturverwaltung
\item Register / Indices erstellen (Bibelstellen)
\end{itemize}
\end{frame}


\begin{frame}
\frametitle{Zusammenarbeit}

\begin{itemize}
\item verschiedene Editoren im Einsatz; Empfehlung TeXStudio
\item Öffentliches github-Repository
\item Jenkins-Server, nächtliche Übersetzung
\item Wechsel auf CI Anbieter (Continous Integration)? -- Was ist das? overleaf?
\end{itemize}
\end{frame}


\begin{frame}
\frametitle{Status}

\begin{itemize}
\item Themenspektrum definiert
\item Zuständigkeiten zugewiesen
\item Konkreter Textzustand sehr verschieden:
  \begin{itemize}
  \item von fast fertig \ldots
  \item \ldots bis nur Stichworte
  \end{itemize}
\item Realistischer Fertigstellungstermin: Erstes Quartal 2019?
\end{itemize}
\end{frame}


\begin{frame}
\frametitle{Herausforderungen}

\begin{itemize}
\item Inhaltliche Schwerpunkte setzen
\item Einheitliches Design bewahren
\item Zeit finden
\end{itemize}
\end{frame}


\begin{frame}
\frametitle{Fazit}

\begin{itemize}
\item Docendo discimus, oder: Ist der Weg das Ziel?
\end{itemize}
\end{frame}


\begin{frame}
\frametitle{Schau mer mal...}

Kurze Live-Demonstration\ldots
\end{frame}


\end{document}