\documentclass[12pt,ngerman]{beamer}
\usepackage[utf8]{inputenc}
\usepackage[T1]{fontenc}
\usepackage{booktabs}
\usepackage{babel}
\usepackage{graphicx}
\usepackage{csquotes}
\usepackage{xcolor}
 
 
\usepackage[sfdefault]{plex-sans}
\usetheme[progressbar=frametitle]{metropolis}           % Use metropolis theme
 
\title{LaTeX für Geisteswissenschaftler - Eine Einführung}
\date{\today}
\author{Thomas Hilarius Meyer, Martin Siever, und Uwe Ziegenhagen}
 
\makeatletter
\setlength{\metropolis@titleseparator@linewidth}{1pt}
\setlength{\metropolis@progressonsectionpage@linewidth}{1pt}
\setlength{\metropolis@progressinheadfoot@linewidth}{1pt}
\makeatother
 
\begin{document}
 
\begin{frame}
	 \maketitle
\end{frame}
 
\begin{frame}
\frametitle{Einleitung}

\begin{itemize}
\item \LaTeX-Literatur für MINT-Fächer ist reichlich vorhanden
\item Geisteswissenschaften etwas \enquote{zu kurz} gekommen
\item Zusammenschluss verschiedener Autoren
\item Ziel: \LaTeX-Einführung für Geisteswissenschaftler, mittelfristig als Buch
\item Zeithorizont: \ldots
\end{itemize}

\end{frame}
 
\begin{frame}
\frametitle{Autoren}

Wer ist beteiligt?

\begin{itemize}
\item 
\item 
\item 
\item 
\item 
\item Dr. Uwe Ziegenhagen, Köln
\end{itemize}

\end{frame}

\begin{frame}
\frametitle{Wie es begann\ldots}


\begin{itemize}
\item 
\item 
\item 
\item 
\item 
\item 
\end{itemize}
\end{frame}

 
\begin{frame}
\frametitle{Themenüberblick}


\begin{itemize}
\item 
\item 
\item 
\item 
\item 
\item 
\end{itemize}
\end{frame}
 
\begin{frame}
\frametitle{Zusammenarbeit}


\begin{itemize}
\item Öffentliches github-Repository
\item Jenkins-Server, nächtliche Übersetzung
\item Wechsel auf CI Anbieter (Continous Integration)
\item 
\item 
\item 
\end{itemize}
\end{frame}
 
 
\begin{frame}
\frametitle{Status}

\begin{itemize}
\item 
\item 
\item 
\item 
\item 
\item 
\end{itemize}
\end{frame}

\begin{frame}
\frametitle{Herausforderungen}


\begin{itemize}
\item Einheitliches Design bewahren
\item Zeit finden
\item 
\item 
\item 
\item 
\end{itemize}
\end{frame}


\begin{frame}
\frametitle{Fazit}


\begin{itemize}
\item 
\item 
\item 
\item 
\item 
\item 
\end{itemize}
\end{frame} 
 
\end{document}