\documentclass[a5paper]{article} 

\usepackage[pass]{geometry}

\usepackage[latin,ngerman]{babel}

\usepackage[%
series={},%keine Apparate aktivieren
noend,%keine Endotenapparate
noeledsec,%keine eledsections et al.
noledgroup%keine ledgroups
]{reledmac}%

\usepackage{reledpar}

\begin{document}

%%% die se\firstlinenumR{2}
\linenumincrementR{2}
\setlength{\columnrulewidth}{0.5pt}
\setlength{\Lcolwidth}{0.425\textwidth}
\setlength{\Rcolwidth}{0.425\textwidth}
\columnsposition{C}

\begin{pages}
\begin{Leftside}
\beginnumbering
\autopar
\selectlanguage{latin}\itshape

\pstart[\section*{Itinerarium Egeriae XXIII}]
Ibi autem ad sanctam ecclesiam nichil aliud est nisi monasteria sine numero uirorum ac mulierum. Nam inueni ibi aliquam amicissimam michi, et cui omnes in oriente testimonium ferebant uitae ipsius, sancta diaconissa nomine Marthana, quam ego aput Ierusolimam noueram, ubi illa gratia orationis ascenderat; haec autem monasteria aputactitum seu uirginum regebat.
\pend

\autopar

... monasteria ergo plurima sunt ibi per ipsum collem et in medio murus ingens, qui includet ecclesiam, in qua est martyrium, quod martyrium satis pulchrum est. Propterea autem murus missus est ad custodiendam ecclesiam propter Hisauros, quia satis mali sunt et frequenter latrunculantur, ne forte conentur aliquid facere circa monasterium, quod ibi est deputatum.

\endnumbering
\end{Leftside}

\begin{Rightside}
\beginnumbering

\pstart[\section*{Itinerarium Egeriae XXIII}]
Gallien als ganzes ist in drei Teile gegliedert,
    deren erster die Belger bewohnen,
    den zweiten die Aquitanier,
    den dritten jene, die in ihrer eigenen Sprache Kelten, in der unseren Gallier genannt werden.
\pend

\autopar

Sie alle unterscheiden sich hinsichtlich der Sprache, 
    der (staatlichen) Einrichtungen und der Gesetze von einander.
    
\endnumbering
\end{Rightside}
\end{pages}

% hier nun alles ausspucken, linke und rechte Seiten
\Pages

\end{document}