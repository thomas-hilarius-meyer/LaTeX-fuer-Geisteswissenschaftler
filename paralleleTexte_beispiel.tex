\documentclass[a5paper]{article} 

\usepackage[pass]{geometry}

\usepackage[latin,ngerman]{babel}

\usepackage[%
series={},%keine Apparate aktivieren
noend,%keine Endotenapparate
noeledsec,%keine eledsections et al.
noledgroup%keine ledgroups
]{reledmac}%

\usepackage{reledpar}

\begin{document}

%%% die se\firstlinenumR{2}
\linenumincrementR{2}
\setlength{\columnrulewidth}{0.5pt}
\setlength{\Lcolwidth}{0.425\textwidth}
\setlength{\Rcolwidth}{0.425\textwidth}
\columnsposition{C}

\begin{pages}
\begin{Leftside}
\beginnumbering
\autopar
\selectlanguage{latin}

Gallia est omnis divisa in partes tres, 
    quarum unam incolunt Belgae,
    aliam Aquitani,
    tertiam, qui ipsorum lingua Celtae, nostra Galli appellantur.

Hi omnes lingua institutis legibus inter se differunt.

\endnumbering
\end{Leftside}

\begin{Rightside}
\beginnumbering
%\pstart
\autopar

Gallien als ganzes ist in drei Teile gegliedert,
    deren erster die Belger bewohnen,
    den zweiten die Aquitanier,
    den dritten jene, die in ihrer eigenen Sprache Kelten, in der unseren Gallier genannt werden.

Sie alle unterscheiden sich hinsichtlich der Sprache, 
    der (staatlichen) Einrichtungen und der Gesetze von einander.
    
\endnumbering
\end{Rightside}
\end{pages}

% hier nun alles ausspucken, linke und rechte Seiten
\Pages

\end{document}