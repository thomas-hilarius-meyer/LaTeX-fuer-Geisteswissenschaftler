% !TeX root = lfgw.tex
\chapter{Textpassagen in nicht-lateinischen Alphabeten einbetten}

Hinweis von Lukas C. Bossert:
Für einzelne Wörter kann man auch \url{http://www.perseus.tufts.edu/hopper/morph} nutzen.
bzw. für antike Texte auch hieraus (\url{http://www.perseus.tufts.edu/hopper/collection?collection=Perseus:collection:Greco-Roman}) kopieren.

\section{Allgemeines: babel, Unicode}

Grundlegend ist der Artikel von Axel Kielhorn in der DTK.%
\footcite{kielhorn:dtk2014}

\section{Eingabe von Unicode-Sonderzeichen}
\label{unicodeeingabe}

\minisec{Eingabe von Unicode-Zeichen in
einem \Program{Emacs} Buffer}
\label{unicodeviaemacs}

\section{Test Block von arktisvogel}

\minisec{Wie arktisvogel die \texttt{lfgw}-Makros anwendet}
\label{cpftestalpha}

Es befriedigt \LaTeX{}-Benutzer die Programme \LuaLaTeX ,
\METAFONT{}, \pdfLaTeX{} und \XeLaTeX{} zu benutzen. Um einen
Kropfnugn zu gestalten, benutzen sie einen Paket wie
\Package{booktabs} oder \paket{graphicx}.
Nach einige %
\cs {biegen\marg {color}}%
\space%
in der %
\cs {biegen\oarg {landscape}\marg {color}}%
\space%
zu bestimmen. Auch zwei %
\cs {biegen\parg {purple}}
\space%
Sachen sind %
\meta {offene Vokale}%
\space%
wichtig.

\minisec{Wie arktisvogel das \paket{keystroke}-Paket anwendet}
\label{cpftestbeta}
Nach einige \cs{biegen\marg{Buenos Aires}} in der
\cs{biegen\oarg{Sao Paulo}} zu bestimmen. Auch zwei
\cs{biegen\parg{Ciudad Mexico}} Sachen sind
\meta{skateboard} wichtig.

\minisec{Wie arktisvogel die \paket{Keystroke}-Paket anwendet}
\label{cpftestbety}

Das Paket \paket{keystroke} bietet die Möglichkeit,
Tastendrücke darzustellen.

\minisec{Almut A.\ Altquist}

\emph{Achtung: zur Zeit die Benutzung der keystroke Paket
liefert nur französische Tasten-Darstellungen!} Das
Paket \paket{keystroke} bietet die Möglichkeit,
Tastendrücke darzustellen. Für ja,
die \hbox{\keystroke{J}-Taste} ist gut.
Wenn sie \Del{} und %
\Ins{}%
und %
\Esc{}%
und %
\Shift{}%
und %
\Ctrl{}%
und %
\Home{}%
und %
\End{}%
und %
\PgUp{}%
und %
\PgDown{}%
und %
\PrtScroll{}%
und %
\Spacebar{}%
und %
\Break{} sehen wollen, geht es in Ordnung.

\minisec{Bertha B.\ Bertzbach}
Der Schriftsteller Markus Arktisvogel
braucht die Symbole %
\Esc{}%
und %
\Shift{}%
und %
\Ctrl{}%
und %
\Alt{}%
und %
\Return{}%
und %
\Spacebar{}%
um alle Probleme zu lösen.

Karl hat drei \Spacebar{}%
und %
\Return{}%
und %
\BSpace{}%
und %
\Tab{}%
und %
\Alt{}%
und %
\AltGr{}%
und %
\NumLock{}%
und %
\UArrow{}%
und %
\DArrow{}%
und %
\LArrow{}%
und %
\RArrow{} vorhanden.

$$\diamondsuit\qquad\diamondsuit\qquad\diamondsuit$$

Zwei Methoden ermöglichen die Eingabe von
Unicode-Zeichen in \Program{Emacs}:

\minisec{Methode 1:\enspace Hexdezimalzahl eingeben}
\Program{Emacs}-Dokumentation benutzt die Schreibweise: %
\texttt{C-x 8}
um den Vorgang zu beschreiben: man druckt die Tastenkombination
\Ctrl$+$\keystroke{X}%
\space%
\keystroke{8}.

In diesem Fall heisst das: %
\Ctrl$+$\keystroke{X}%
\space%
\keystroke{8}%
\Return{}%
\meta{Unicode Codepoint}%
\Return . ermöglicht die direkte Eingabe einer
Unicode Zeichen.

Um einem \enquote{Capitulum} einzugeben: %
\hbox{\Ctrl$+$\keystroke{X}}%
\keystroke{8}%
\Return{}%
\keystroke{2}%
\keystroke{E}%
\keystroke{3}%
\keystroke{F}%
\space%
\Return{}%

% Die Unicode Codepoint 2021 ist: Double Dagger, und sieht so aus: ``‡''
% Die Unicode Codepoint 2045 ist: Left Square Bracket With Quill, und sieht so aus: ``⁅''
% Die Unicode Codepoint 2046 ist: Right Square Bracket With Quill, und sieht so aus: ``⁆''
% Die Unicode Codepoint 2e3f ist: Capitulum, und sieht so aus: ``⸿''
% Die Unicode Codepoint 02ad ist: Latin Letter Bidental Percussive, und sieht so aus: ``ʭ''
% Die Unicode Codepoint 169b ist: Ogham Feather Mark, und sieht so aus: ``᚛''
% Die Unicode Codepoint 169c ist: Ogham Reversed Feather Mark, und sieht so aus: ``᚜''
%
\minisec{Methode 2:\enspace \Program{Emacs}-Aliase Benutzen}

\Program{Emacs} hat eine lange Liste vorgefertigte Aliase%
\footnote{Ein \enquote{Alias} ist eine Aufruf, mit der
mehrere Funktionen, durch einen neuen Befehl ersetzt werden kann.
Es wird benutzt, um Zeit zu sparen und weniger zu tippen.}%
um oft benutzte Unicode Zeichen einzugeben. Zuerst, mit nur
drei Tastendrücke: um eine \enquote{no-break space} einzugeben: %
\hbox{\Ctrl$+$\keystroke{X}}%
\keystroke{8}%
\Spacebar .

Einige Aliase sind fertig nach drei Tastendrücke. Andere
benötigen vier Tastendrücke. Ich möchte das Wort \enquote{Français}
eingeben. Mit %
\hbox{\Ctrl$+$\keystroke{X}}%
\keystroke{,}%
\keystroke{C}%
\space%
kann ich einen \enquote{c} mit eine Cedille eingeben.

Ich möchte das Wort \enquote{mañana} eingeben. Mit %
\hbox{\Ctrl$+$\keystroke{X}}%
\keystroke{\char126}%
\keystroke{N}%
\space%
kann ich einen \enquote{n} mit einen Tilde eingeben.

Es gibt auch Aliase für drei verschiedene Bruchzahlen. Aber
\TeX{}-Benutzer haben nie Probleme damit gehabt, Bruchzahlen
zu setzen.

\minisec{Eingabe von Unicode-Zeichen in \Program{gedit}}
\label{unicodeviagedit}

\Program{gedit} ermöglicht auch die direket Eingabe einer Unicode
Zeichen durch Tastendruck-Folge %
\hbox{\Shift$+$\Ctrl$+$\keystroke{U}}%
\meta{Unicode Codepoint}%
\Return{}%
\space%
ermöglicht die direkte Eingabe einer Unicode
Zeichen.

Mit %
\hbox{\Shift$+$\Ctrl$+$\keystroke{U}}%
\keystroke{2}%
\keystroke{0}%
\keystroke{2}%
\keystroke{2}%
\Return{}%
\space%
kann ich einen \enquote{Bullet} eingeben.

%% \minisec{Eingabe von Unicode-Zeichen in \Program{nano}}
%% \label{unicodevianano}

%% \Program{GNU nano} ermöglicht auch die direket Eingabe einer
%% Unicode Zeichen.

%% \begin{foreigndisplayquote}{english}
%% Entering Text

%% nano is a “modeless” editor. This means that all keystrokes, with the exception of Control and Meta sequences, enter text into the file being edited.

%% Characters not present on the keyboard can be entered in two ways:

%% For characters with a single-byte code, pressing the Esc key twice and then typing a three-digit decimal number (from 000 to 255) will enter the character with the corresponding value.

%% For any possible character, pressing M-V (Alt+V) and then typing a six-digit hexadecimal number (starting with 0 or 1) will enter the character with the corresponding Unicode value.
%% \end{foreigndisplayquote}
%%
%% \textbf{Texteingabe}

%% \Program{nano} erlaubt die Eingabe jede mögliche Zeichen
%% (ausser Meta-Sequenzen und Steuerung-Sequenzen) direkt in
%% einem Textdatei einzugeben.

%% Es gibt eine Methode um Zeichen die in dem Bereich 0$_{10}$
%% bis zum 255$_{10}$ (hex FF$_{16}$)~-- ein Byte Zeichen~--
%% einzugeben. Ein zweite Methode erlaubt die Eingabe von zwei
%% Byte Zeichen.

%% Methode 1:~zweimal die %
%% \Esc\Esc{}$\langle$\texttt {dreistellige Dezimalzahl}$\rangle$

%% Methode 2:~\Alt$+$\keystroke{V}$\langle$\texttt {sechsstellige Hexadezimalzahl}$\rangle$%

\minisec{Eingabe direkt mit der Tastatur}

Lohnend besonders bei den modernen Fremdsprachen: Anschaffen einer Tastatur in der jeweiligen
Sprache.

Problem im altphilologischen Bereich: Akzente und (bei Hebräisch) Vokalzeichen sind heute
nicht mehr verwendet und fehlen deshalb auf den heutigen Tastaturen.

\minisec{Auswahl über Maus-gestützte tools}
KCharselect

\begin{figure}
 \includegraphics[width=\textwidth]{kcharselect}
 \caption{Mit dem KDE-Programm KCharselect kann man Unicode-Buchstaben immerhin mit der Maus
 auswählen.}
\end{figure}

\minisec{Eingabe über die Zeichennummer}

Vgl.\ Abschnitt \ref{utf8codes} auf Seite \pageref{utf8codes}.

\section{Griechisch}
\index{Griechisch}

Es gibt (mindestens...) drei verschiedene Möglichkeiten, griechische Textpassagen einzubinden;
je nach Aufgabenbreich sind diese verschieden gut geeignet:


\minisec{Möglichkeit I: Einzelbuchstaben}

Die erste Möglichkeit besteht in der Eingabe m.\,H. der Einzelnbuchstaben-Symbole des Paketes
\paket{textgreek}; das Verfahren ist in Abscnitt \ref{griechEinzelbuchstaben},
S.~\pageref{griechEinzelbuchstaben}, beschrieben.

Dieses Verfahren eignet sich eigentlich nur für ganz kurze Einsprengsel im Umfang von
einzelnen Zeichen bis maximal ca.\ einem Wort.


\minisec{Möglichkeit II: Transkription mit \paket{ibycus}}

Die zweite Methode eignet sich demgegenüber hervorragend, wenn es darum geht auch etwas
längere (alt-)griechische Passagen in ein ansonsten deutschsprachiges Dokument~-- m.\,H. eines
deutschen Computers mit QWERTZ-Tastatur~-- einzugeben.

Dabei wird für die griechischen Buchstaben eine Art Transkription in lateinischen Buchstaben
vorgenommen, die sich bei etwas Übung sehr leicht benutzen lässt.

Das Paket \paket{ibycus}%
\footnote{Paketdokumentation: texdoc ibycus-babel}
definiert dabei eine Art Pseudosprache für \paket{babel},
in der Dokumenpräambel ist also anzugeben:

\begin{lstlisting}
 \usepackage[ibycus,ngerman]{babel}
\end{lstlisting}

Im Dokument stehen dann eine Umgebung \lstinline/ibycus/ sowie ein Befehl \lstinline/\ibygr{}/
zur Verfügung:


Das wird zu:

 \begin{ibycus}
  (Hrodo'tou Qouri’ou i(stori’hs a)po’decis h(’de,
  ...
  h(‘n ai)ti’hn e)pole’mhsan a)llh’loisi.
  \end{ibycus}

  Und jetzt ein einzelnes griechisches Wort~-- \ibygr{a)rxai=a gra’mmata}~-- im Text.





\minisec{Möglichkeit III: Direkteingabe von Unicode-Text}

Die dritte Möglichkeit besteht darin, die griechischen Schriftzeichen direkt als Unicode-Zeichen
in das \LaTeX -Dokument einzufügen. Dazu reicht es aus, dem Paket \paket{babel} die
(alt-)griechische Sprache als zusätzliche Option anzugeben:
\lstinline/\usepackage[polutonikogreek,ngerman]{babel}/.
\footnote{Achtung! Bei den meisten \LaTeX -Installationen im Rahmen von Linux-Distributionen muss
man Pakete nachinstallieren!}
Damit \LaTeX{} den griechischen Font richtig ansprechen kann, braucht auch das Paket \paket{fontenc}
eine Modifikation: \lstinline/\usepackage[OT1,T1]{fontenc}/.

Dann kann mit \lstinline/\selectlanguage{polutonikogreek}/ auf Griechisch umgeschaltet werden.

Der folgende Bibelvers wurde aus dem Internet unverändert in das \LaTeX -Dokument
kopiert. Derzeit werden nicht alle Zeichen in meinem KDE-Editor (kile) korrekt dargestellt (vgl. Abb.);
dennoch gibt \LaTeX{} alle Zeichen richtig wider:

\begin{figure}
 \includegraphics[width=\textwidth]{ersatzdarstellung}
 \caption{KDE kann nicht alle aus dem Internet kopierten Unicode-Zeichen korrekt darstellen;
 Buchstaben mit Akzent und Spiritus bekommen eine (nichtssagende) Ersatzdarstellung!}
\end{figure}


\begin{otherlanguage}{polutonikogreek}
Οὕτως γὰρ ἠγάπησεν ὁ Θεὸς τὸν κόσμον, ὥστε τὸν Υἱὸν τὸν μονογενῆ ἔδωκεν,
ἵνα πᾶς ὁ πιστεύων εἰς αὐτὸν μὴ ἀπόληται ἀλλ’ ἔχῃ ζωὴν αἰώνιον.
\end{otherlanguage}
(Joh. 3,16)

Diese Methode dürfte sich in der Praxis am ehesten eignen, wenn bereits ein Textcorpus
(z.\,B. via Internet) ediert und Unicode-Erfasst vorliegt, und Textpartien ins eigene Dokument
ohne Veränderungen übernommen werden sollen~-- so wie z.\,B. bei Zitaten aus der Bibel oder
der klassischen Literatur.

Will man selbst (alt-)griechische Textpassagen schreiben, scheint das Ibykus-Verfahren
wesentlich angenehmer und effizienter.


\section{Hebräisch}
\index{Hebräisch}

Das Paket \paket{cjhebrew} von Christian Justen eignet sich hervorragend für
Theologen und Orientalisten, denn es erlaubt auch die Vokalisierung sowie die Verwendung
von Akzenten. Auch das Problem der rechts- bzw. linksläufigen Schriften wird sehr
Anwenderfreundlich~-- im Sinne von Anwendern an einem westlichen PC mit ansonsten
lateinischen Alphabet~-- gelöst:

Nach \lstinline/\usepackage{cjhebrew}/%
\footnote{Eine besondere Angabe z.\,B. bei \paket{babel} ist nicht nötig.}
steht v.\,a. der Befehl \lstinline/\cjRL{}/
-- für hebräische Einzelworte im laufenden Absatz~-- sowie die
Umgebung \lstinline/cjhebrew/
-- für ganze Absätze in Hebräisch~-- zur Verfügung.

Beide drehen die Folge der Schriftzeichen automatisch um, so dass hebräische Textpartien
ganz gewohnt eingegeben werden können. Dabei sind für die einzelnen Zeichen sehr leicht
merkbare Codes definiert worden:

Der Beginn des biblischen Buches Genesis~-- \enquote{Im Anfang schuf Gott Himmel und Erde...} --
wird so eingegeben:

\begin{lstlisting}
\begin{cjhebrew}
 b*e:re'+siyt b*ArA' 'E:lohim 'et ha+s*amayim w:'et hA'ArE.s;
\end{cjhebrew}
\end{lstlisting}

Mit etwas Eingewöhnung arbeitet man mit diesem System \emph{wesentlich} schneller und
angenehmer, als wenn man etwa die Unicode-Zeichen mit Hilfe eines grafischen Auswahlwerkzeuges
einzeln auswählt...

Die Bearbeitung mit \LaTeX{} ergibt:

\begin{cjhebrew}
 b*e:re'+siyt b*ArA' 'E:lohim 'et ha+s*amayim w:'et hA'ArE.s;
\end{cjhebrew}

Wegen der Einzelheiten der Zeichen-, Vokal-, Akzent- und Satzzeichen-Codierung (inkl. z.\,B.
Mem finalis) sei auf die Paketdokumentation verwiesen.

\section{Russisch}
\index{Kyrillisch}
\index{Russisch}

\section{Koptisch}
\index{Koptisch}

\section{Altkirchenslawisch}
\index{Altkirchenslawisch}

\paket{churchslavonic}


\section{Arabisch}
\index{Arabisch}

Ist \paket{arabtex} noch aktuell?

\section{Hieroglyphen}
\index{Hieroglyphen}
\index{Ägyptologie}

\paket{hieroglf}~-- \enquote{The Poor Man’s Hieroglyphic Font}


 \pmglyph{K:l-i-o-p-a-d:r-a}


\section{Keilschrift}
\index{Keilschrift}

\paket{archaic} und andere...

\section{Runen}
\index{Runen}

Das Paket \paket{runic} von Peter Wilson stellt die Runen des sog.\ Futhark-Alphabets bereit.
Nach \lstinline/\usepackage{runic}/ gibt es den Befehl \lstinline/\textfut{}/, der
die Runen ausgibt. Dies ist die Transkription:

\begin{center}
\begin{tabular}{ll}
 \lstinline/F/ &	\textfut{F} \\
 \lstinline/U/ &	\textfut{U} \\
 \lstinline/\Fthorn/ &	\textfut{\Fthorn} \\
 \lstinline/A/ &	\textfut{A} \\
 \lstinline/R/ &	\textfut{R} \\
 \lstinline/K/ &	\textfut{K} \\
 \lstinline/G/ &	\textfut{G} \\
 \lstinline/W/ &	\textfut{W} \\
 \lstinline/H/ &	\textfut{H} \\
 \lstinline/N/ &	\textfut{N} \\
 \lstinline/I/ &	\textfut{I} \\
 \lstinline/J/ &	\textfut{J} \\
 \lstinline/Y/ &	\textfut{Y} \\
 \lstinline/P/ &	\textfut{P} \\
 \lstinline/X/ &	\textfut{X} \\
 \lstinline/S/ &	\textfut{S} \\
 \lstinline/T/ &	\textfut{T} \\
 \lstinline/B/ &	\textfut{B} \\
 \lstinline/E/ &	\textfut{E} \\
 \lstinline/M/ &	\textfut{M} \\
 \lstinline/L/ &	\textfut{L} \\
 \lstinline/\Fng/ &	\textfut{\Fng} \\
 \lstinline/D/ &	\textfut{D} \\
 \lstinline/O/ &	\textfut{O} \\
 \lstinline/:/ &	\textfut{:} \\
 \end{tabular}
 \end{center}


\section{Phonetische Alphabete}


\section{Kurzschriften}
\index{Kurzschrift}
\index{Schnellschrift}

\minisec{Tironische Noten}
\index{Tironische Noten}


\minisec{Deutsche Einheits-Kurzschrift}
DEK\footcite{sarman:dtk2009/1}
\index{Deutsche Einheits-Kurzschrift}
\index{DEK}


\minisec{Pitman-Kurzschrift}
\index{Pitman}
