
\chapter{Diagramme zeichnen}
allgemeine vs. spezielle Lösungen

\section{Der unsaubere Weg: Open Office und co. benutzen}

Einbinden der fertigen Abb. als Grafik


\section{Die beiden wichtigsten Pakete: pstricks und tikz}

\minisec{pstricks}
\paket{pstricks}

\minisec{tikz}
\paket{tikz}

\section{Linguistische Strukturen}
\index{Linguistik} \index{x-bar-Schema} \index{Phrasenstruktur}
\paket{covington}
\footcite{roemer:dtk2008}
\footcite{roemer:dtk2016}


\minisec{Baumdiagramme}
\index{Baumdiagramm} \index{Stemma}

Das Paket \paket{forest} von Saso Zivanovic erlaubt einfache und sehr komplexe Baumsstrukturen:

\begin{LTXexample}
 \begin{forest}
  [VP
    [DP]
    [V'
      [V]
      [DP]
    ]
  ]
\end{forest}
\end{LTXexample}

\section{Ausgewählte Anwendungen}


\minisec{Zeitschienen}


\minisec{Stammbäume}


\minisec{Statistiken visualisieren}

Torten- und Balkendiagramme... 

\paket{datatool} ?


