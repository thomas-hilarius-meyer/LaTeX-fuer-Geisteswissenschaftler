% !TeX root = lfgw.tex
%\chapter{Texte parallel setzen}

\newcommand\reledpar{\mbox{\Package{reledpar}}\xspace}

\section{Vertikal parallelisierte Texte}
\autor{Philipp Pilhofer}

\DefineShortVerb{\+}

Es gibt mehrere \LaTeX-Pakete, mit denen man Texte spalten- oder seitenweise parallel setzen kann. Hier wird das Paket \reledpar vorgestellt, weil es gut funktioniert, viele Möglichkeiten bietet und auch aktuell noch gewartet wird. Dieses Paket arbeitet eng mit dem im vorangegangenen Kapitel vorgestellten \reledmac-Paket zusammen; daher werden einige der hier besprochenen Befehle und Einstellungen in der \reledmac-Dokumentation erklärt.

Um Texte vertikal zu parallelisieren, ist nicht viel zu tun. Ich gehe hier erst auf die spaltenweise Parallelisierung ein, danach komme ich zur seitenweisen Parallelisierung.


\subsection{Spaltenweise parallelisierte Texte}

Wie schon für \reledmac gezeigt,%
\footnote{Im folgenden beziehe ich mich öfters auf den Abschnitt zu den \reledmac-Grundlagen, \cref{apparate-grundlagen-anfang}, \cpagerefrange{apparate-grundlagen-anfang}{apparate-grundlagen-ende}.}
muß der Text für die linken und rechten Spalten jeweils mit +\beginnumbering+ und +\endnumbering+ eingefasst werden.%

Der Einfachheit halber lassen wir hier die Paragraphen-Einteilung durch +\autopar+ erledigen;
die Paragraphen werden dann automatisch erkannt und jeweils auf derselben Höhe begonnen. Damit dies auch funktioniert, müssen vor und nach den Paragraphen Leerzeilen stehen.

Die Textabschnitte für die linke Spalte müssen nun noch in eine +Leftside+-Umgebung eingefügt werden, analog die rechte Spalte in eine +Rightside+-Umgebung.
Diese beiden Umgebungen selbst müssen von einer +pairs+-Umgebung (man beachte die jeweilige Groß-/Kleinschreibung) umfaßt werden.

Die Spalten werden nun mit +\Columns+ ausgegeben. Hier ein Beispiel:


%das hier steht hier nur, um die im vorigen Kapitel vorgenommenen Einstellungen zurückzunehmen
\firstlinenum{5}
\linenumincrement{5}

\begin{lfgwcode}{}
\begin{pairs}
\begin{Leftside}
\beginnumbering
\autopar
\selectlanguage{latin}

Gallia est omnis divisa in partes tres, 
    quarum unam incolunt Belgae,
    aliam Aquitani,
    tertiam, qui ipsorum lingua Celtae, nostra Galli appellantur.

Hi omnes lingua institutis legibus inter se differunt.

\endnumbering
\end{Leftside}

\begin{Rightside}
\beginnumbering
%\pstart
\autopar

Gallien als ganzes ist in drei Teile gegliedert,
    deren erster die Belger bewohnen,
    den zweiten die Aquitanier,
    den dritten jene, die in ihrer eigenen Sprache Kelten, in der unseren Gallier genannt werden.

Sie alle unterscheiden sich hinsichtlich der Sprache, 
    der (staatlichen) Einrichtungen und der Gesetze von einander.
    
\endnumbering
\end{Rightside}
\end{pairs}

\Columns
\end{lfgwcode}

Dieser Code wird so ausgegeben:

%damit die columns in die bsp-box passen:
\columnsposition{C}
\setlength{\Lcolwidth}{0.425\textwidth}
\setlength{\Rcolwidth}{0.425\textwidth}

\begin{lfgwprint}{}
\begin{pairs}
\begin{Leftside}
\beginnumbering
\autopar
\selectlanguage{latin}

Gallia est omnis divisa in partes tres, 
    quarum unam incolunt Belgae,
    aliam Aquitani,
    tertiam, qui ipsorum lingua Celtae, nostra Galli appellantur.

Hi omnes lingua institutis legibus inter se differunt.

\endnumbering
\end{Leftside}

\begin{Rightside}
\beginnumbering
%\pstart
\autopar

Gallien als ganzes ist in drei Teile gegliedert,
    deren erster die Belger bewohnen,
    den zweiten die Aquitanier,
    den dritten jene, die in ihrer eigenen Sprache Kelten, in der unseren Gallier genannt werden.

Sie alle unterscheiden sich hinsichtlich der Sprache, 
    der (staatlichen) Einrichtungen und der Gesetze von einander.
    
\endnumbering
\end{Rightside}
\end{pairs}

\Columns
\end{lfgwprint}

Analog zu den oben für \reledmac vorgestellten Optionen gibt es nun für die rechten Spalten analoge Befehle. Für die Zeilenzähler sind dies beispielsweise +\firstlinenumR{<num>}+ und +\linenumincrementR{<num>}+. Hinzu kommen die für Spaltensatz erwartbaren Möglichkeiten: Mit der Länge +\columnrulewidth+ kann man einen Trennungsstrich einführen, mit +\Lcolwidth+ und +\Rcolwidth+ die Spaltenbreite verändern. Mit dem Befehl +\columnsposition{<pos>}+ läßt sich die Orientierung der Spalten ändern: Standardmäßig orientieren sie sich am rechten Rand, mit +Ĺ+ orientieren sie sich am linken Rand, mit +C+ sind sie zentriert. Alle Längen müssen mit +\setlength{<name>}{<laenge>}+ angepaßt werden, hier ein Beispiel für die vorgestellten Befehle: 

\begin{lfgwcode}{}
\firstlinenumR{2}
\linenumincrementR{2}
\setlength{\columnrulewidth}{0.5pt}
\setlength{\Lcolwidth}{0.425\textwidth}
\setlength{\Rcolwidth}{0.425\textwidth}
\columnsposition{C}
\end{lfgwcode}

Bemerkenswert ist vielleicht noch die Paket-Option +movecolumnspositiononrightpage+ für Bücher: Ist diese gesetzt, werden die \enquote{rechten} Spalten auf rechten Seiten links gesetzt, d.h. die \enquote{rechten} Spalten stehen immer auf der Innenseite. Weitere Optionen bietet das \reledpar-Handbuch.


\subsection{Seitenweise parallelisierte Texte}

Nach der Lektüre des vorangegangenen Abschnittes ist es sehr einfach, Texte seitenweise zu parallelisieren, was sich vor allem bei Editionen mit Übersetzung anbietet: So kann auf den linken Seiten der Originaltext gedruckt werden, auf der rechten Seite die Übersetzung; die einzelnen Paragraphen stehen, wie im spaltenweise parallelisierten Satz, auf derselben Höhe.

Die Texte müssen, wie im Spaltensatz, in +\beginnumbering+/+\endnumbering+ und +\autopar+ sowie die +Leftside+-/+Rightside+-Umgebung eingefaßt werden. Anders sind hier nur die Namen der äußeren Umgebung, die heißt +pages+, und der Befehl, um alles auszugeben, der lautet +\Pages+

... und immernoch nicht fertig ...

%%\begin{lfgwprint}
%\begin{pages}
%\begin{Leftside}
%\beginnumbering
%der Text für die linken Spalten/Seiten
%\endnumbering
%\end{Leftside}
%
%\begin{Rightside}
%\beginnumbering
%der Text für die rechten Spalten/Seiten
%\endnumbering
%\end{Rightside}
%\end{pages}
%
%\Pages
%%\end{lfgwprint}
\UndefineShortVerb{\+}

%%%das war früher einmal:
%Das Paket \paket{parallel} von Matthias Eckermann stellt eine Umgebung 
%\lstinline/Parallel/ zur Verfügung, in der mit den Befehlen 
%\lstinline/\ParallelLText{...}/ und 
%\lstinline/\ParallelRText{...}/ die jeweils links- bzw. rechtsstehenden Texte
%angegeben werden können.
%
%Durch den Befehl \lstinline/\ParallelPar/ wird ein neuer \enquote{Doppelabsatz} begonnen.
%
%\begin{lstlisting}
% \linenumbers
% \begin{Parallel}{.45\textwidth}{.45\textwidth}
%  \ParallelLText{Gallia est omnis divisa in partes tres ...}
%  \ParallelRText{Gallien als ganzes ist in drei Teile...}
%  \ParallelPar
%  \ParallelLText{Hi omnes...}
%  \ParallelRText{Sie alle...}
% \end{Parallel}
%\end{lstlisting}
%
%Beim Erzeugen der \lstinline/Parallel/-Umgebung ist für die linke und rechte Seite ihre
%jeweilige Breite anzugeben; das kann als absolute Angabe (in mm oder cm) oder als 
%Bezugnahme zu einem Wert wie \lstinline/\textwidth/ (der Breite des aktuellen 
%Satzspiegels) erfolgen.
%
%Das obige Beispiel erzeugt folgende Ausgabe:
%\bigskip 
%
%\linenumbers
%\begin{Parallel}{.45\textwidth}{.45\textwidth}
%  \ParallelLText{Gallia est omnis divisa in partes tres, 
%    quarum unam incolunt Belgae,
%    aliam Aquitani,
%    tertiam, qui ipsorum lingua Celtae, nostra Galli appellantur.}
%  \ParallelRText{Gallien als ganzes ist in drei Teile gegliedert,
%    deren erster die Belger bewohnen,
%    den zweiten die Aquitanier,
%    den dritten jene, die in ihrer eigenen Sprache Kelten, in der unseren Gallier genannt werden.}
%  \ParallelPar
%  \ParallelLText{Hi omnes lingua institutis legibus inter se differunt.}
%  \ParallelRText{Sie alle unterscheiden sich hinsichtlich der Sprache, 
%    der (staatlichen) Einrichtungen und der Gesetze von einander.}
% \end{Parallel}
%\nolinenumbers
%
%(Zu der Zeilennummerierung vgl. Abschnitt \ref{zeilennummer} auf S.~\pageref{zeilennummer}.)
%
%\minisec{Das Paket \paket{paracol}}
%
%\paket{paracol}