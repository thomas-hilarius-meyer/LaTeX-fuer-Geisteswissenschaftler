%% lfgw: 2.21 Linguistische Beispiele
\subsubsection{Einordnung}

Konstituentenanalysen von Wörtern und Sätzen vorzunehmen, sind in der Linguistik vom amerikanischen Strukturalismus entwickelte Verfahren der Segmentierung zur Ermittlung von syntaktischen Konstituentenhierarchien 
und -kategorien, die für die Beschreibung und Analyse von Sprachen relevant sind. 
Die Ergebnisse der Zerlegungen werden in Baumstrukturen (s. Beispiel~\ref{forestexample:1}) und  Klammerschreibungen (s. Beispiel~\ref{forestexample:2})
veranschaulicht. Noam Chomsky, der Kopf der Generativen Grammatik, zeigt in Form der X-bar-Theorie
z.\,B. in \enquote{Rules and Representations} die Beziehung zwischen Tense und dem COMP  folgendermaßen auf (S.\,170f.)

\begin{lfgwexample}{label={forestexample:1}}
\begin{forest}
[S´
  [COMP [$ \begin{Bmatrix}\emph{that} \\ \emph{for} \end{Bmatrix} $, align=center, base=bottom]]
  [S
   [NP [\emph{John}, tier=word]]
   [ $ \begin{Bmatrix} \emph{Tense}\\ \emph{to} \end{Bmatrix} $, align=center, base=bottom]
   [VP [\emph{leave}, tier=word]]
  ] ]
\end{forest}
\end{lfgwexample}

\begin{lfgwprint}{label={forestexample:2}}
\begin{center}
it is unclear $[_\text{S´} [_{\text{COMP}} \enspace \textnormal{what} ]
 [_\text{S}  \enspace  \textnormal{NP} \enspace \begin{Bmatrix} \emph{Tense}\\ \emph{to} \end{Bmatrix} \enspace \textnormal{VP}] ]$
\end{center}
\end{lfgwprint}

\lstset{frame=single}
\begin{lstlisting}
\begin{center}
it is unclear $[_\text{S´} [_{\text{COMP}} \enspace \textnormal{what} ]
 [_\text{S}  \enspace  \textnormal{NP} \enspace \begin{Bmatrix}{c}\emph{Tense}\\ \emph{to} \end{Bmatrix} \enspace \textnormal{VP}] ]$
\end{center}
\end{lstlisting}


\subsubsection{Baumstrukturen mit \texttt{forest}}

Sa\v{s}o \v{Z}ivanovi\'{c}, der Autor von \Package{forest}, schreibt in
\cite[3]{zivanovic:forest},
bezüglich Forest: \\
"`I believe, the most flexible tree typesetting package for \LaTeX\ you can get."'

Auf jeden Fall ist es einfacher als andere Pakete\footnote{Zu anderen
einschlägigen Paketen siehe \cite{roemer:dtk2016}.} zu handhaben, da es die Spreizung
der Äste entsprechend der Label an den Astenden selbstständig anpasst. Dabei bemüht es sich,
um schlanke Bäume. Auch komplexe Strukturen führen nicht zu Problemen. Durch die Anbindung
an PGF/TikZ können Ausschmückungen vorgenommen werden.

\minisec{Funktionsweise}

Das Paket wird mit \verb|\usepackage{forest}| in der Präambel geladen. Spezielle Bibliotheken
können mit ihrem Namen in der Regel als fakultatives Argument hinzugefügt werden: 
\verb|\usepackage[library name]{forest}|; beispielsweise diejenige für linguistische Strukturen
mit \verb|\usepackage[linguistics]{forest}|. Mit dieser Option werden die Strukturbäume
standardmäßig linksverzweigt.

Phonologische Strukturen (wie in Beispiel~\ref{forestexample:3}) sind in der \texttt{linguistik}-Bibliothek
in dem \texttt{GP1}-Stil angelegt. Der Stilname wird vor die erste sich öffnende Klammer des Baums 
geschrieben\footnote{\cite[Beispiel\,6]{zivanovic:forest}}.

\begin{lfgwexample}{label={forestexample:3}}
\begin{forest} GP1 [
  [O[x[f]][x[r]]]
  [R[N[x[o]]][x[s]]]
  [O[x[t]]]
  [R[N[x]]]
]
\end{forest}
\end{lfgwexample}

Die Baumstrukturen werden in die \texttt{forest-Umgebung} oder in den Befehl \verb|Forest*{   }|
mittels Umklammerungen der Konstituenten eingefügt. Die Kinder eines Knotens befinden sich 
innerhalb des Mutterknotens.\footnote{Den linguistischen Konstruktionen liegt die Generative
Grammatik zugrunde. So werden bei den Barstufen der Kategorien die Markierungen rechts ausgedruckt,
obwohl sie links eingegeben werden (siehe Beispiel~\ref{forestexample:4}).}

\begin{lfgwexample}{label={forestexample:4}}
\begin{minipage}{.3\linewidth}
\begin{forest}
[VP
  [DP]
  [V’
   [V]
  [DP]
  ]
]
\end{forest}
\end{minipage} \quad
\begin{minipage}{.2\linewidth}
\Forest*{
[VP
  [DP]
  [V’
   [V]
  [DP]
  ]
]
}
\end{minipage}
\end{lfgwexample}

Als letzte Labels können auch natürlichsprachliche Ausdrücke (\verb|tier=word|) eingefügt werden, 
die dann entsprechend den Konventionen kursiv geschrieben werden können, mit dem entsprechenden
Befehl. Wie im folgenden Beispiel (\ref{forestexample:5}) beim obersten Knoten zusehen ist, 
können Labels auch ohne Kantenverbindungen untereinander
geschrieben werden (\verb|align=center, base=bottom|).


\begin{lfgwexample}{label={forestexample:5}}
\begin{forest}
where n children=0{tier=word}{}
[Wort\\Stamm, align=center, base=bottom
[BM [\emph{Not}, tier=word]]
[Stamm
[Stamm [WBM [Präf [\emph{auf}]]] 
       [Wu [BM$'$ F [\emph{nahm e}]]]]
[WBM [Suff [\O]]]]       
]
\end{forest}
\end{lfgwexample}

Die Linien zu den Konstituenten können auch gestrichelt, gepunktet oder als Pfeile
gesetzt werden (Beispiel~\ref{forestexample:6}).

\begin{lfgwexample}{label={forestexample:6}}
\begin{forest} for tree={grow'=0,l=2cm,anchor=west,child anchor=west},
[Äste,rotate=90
  [normale Linie,edge label={node[midway,left,
        font=\scriptsize]{default}}]
  [keine Linie,no edge]
  [gepunktete Linie,edge=dotted]
  [gestrichelte Linie,edge=dashed]
  [gestrichelte rote Linie,edge={dashed,red}]
  [Pfeil, edge={->, very thick}]
]
\end{forest}
\end{lfgwexample}

Die Bibliothek \texttt{edges} ermöglicht verzweigte Kanten/Äste bei horizontal wachsenden
Strukturen. Sie wird mit \verb|\useforestlibrary{edges}| geladen. Man kann damit beispielsweise
Begriffshierarchien aufzeigen (wie in \ref{forestexample:7}, \ref{forestexample:8}).

\begin{lfgwcode}{label={forestexample:7}}
\documentclass[border=5pt,tikz]{standalone}
\usepackage{xltxtra}
\usepackage{forest}
\useforestlibrary{edges}
\begin{document}
\begin{forest}
 for tree={grow'=1,draw},
forked edges,
[Lebewesen
  [Menschen]
  [Tiere
    [Haustiere]
    [Raubtiere]
  ]  
  [Pflanzen]
]
\end{forest}
\end{document
\end{lfgwcode}

\begin{lfgwprint}{label={forestexample:8}}
\begin{forest}
 for tree={grow'=1,draw},
forked edges,
[Lebewesen
  [Menschen]
  [Tiere
    [Haustiere]
    [Raubtiere]
  ]  
  [Pflanzen]
]
\end{forest}
\end{lfgwprint}


\minisec{Modifizierungen}
%TODO: forestexample:10 existiert nicht mehr => Verweis nach forestexample:9 gelöscht
In die Strukturen können zusätzliche Informationen mit den Mitteln von TikZ
eingebracht werden (wie in Beispiel~\ref{forestexample:9}). Dies ist allerdings nicht so einfach,
wie die Strukturbäume zu setzen, da es Kenntnisse des komplexen PGF/TikZ verlangt.

\begin{lfgwcode}{label={forestexample:9}}
\begin{forest}
[Nomen,name=kompositum
[Adjektiv [\emph{Rot}, tier=word]]
[Nomen,name=kopf
[Nomen [\emph{kehl}, tier=word]]
[Suffix [\emph{chen}, tier=word]]
]
]
\draw[->,red,dashed,very thick] (kopf) -| node[near start,below] {+ N} (kompositum);
\node at (current bounding box.south)
[below=1ex,draw,fill=yellow, ellipse]
{\emph{Abbildung: Ein Possessivkompositum}};
\end{forest}
\end{lfgwcode}

\begin{lfgwprint}{}
\begin{forest}
[Nomen,name=kompositum
[Adjektiv [\emph{Rot}, tier=word]]
[Nomen,name=kopf
[Nomen [\emph{kehl}, tier=word]]
[Suffix [\emph{chen}, tier=word]]
]
]
\draw[->,red,dashed,very thick] (kopf) -| node[near start,below] {+ N} (kompositum);
\node at (current bounding box.south)
[below=1ex,draw,fill=yellow, ellipse]
{\emph{Abbildung: Ein Possessivkompositum}};
\end{forest}
\end{lfgwprint}

FOREST positioniert die Knoten mittels einem rekursiven Algorithmus (genauer siehe 
\cite[Kap.\,2.4]{zivanovic:forest}). Manuelle Abänderungen sind natürlich möglich. Beispielsweise
kann mit der Option 
\texttt{s sep} der Abstand zwischen den Teilsträngen, die aus einer Wurzel kommen, 
gesteuert werden; vgl. \ref{forestexample:11} und
\ref{forestexample:12}).\footnote{\cite[Beispiel\,27]{zivanovic:forest}.}


\begin{lfgwexample}{label={forestexample:11}}
\begin{forest}
for tree={s sep=(3-level)*2mm}
[CP, for tree=draw
[DP, for tree={fill=green},l*=3
[D][NP]]
[TP,for tree={fill=yellow}
[T][VP[DP][V’[V][DP]]]]
]
\end{forest}
\end{lfgwexample}

\begin{lfgwexample}{label={forestexample:12}}
\begin{forest}
for tree={s sep=(3-level)*6mm}
[CP, for tree=draw
[DP, for tree={fill=green},l*=3
[D][NP]]
[TP,for tree={fill=yellow}
[T][VP[DP][V’[V][DP]]]]
]
\end{forest}
\end{lfgwexample}

%\measurexdistance{(!11.south east)}{(!12.south west)}{+(0,-5mm)}{below}
%\path(md2)-|coordinate(md)(!221.south east);
%\measurexdistance{(!221.south east)}{(!222.south west)}{(md)}{below}
%\measurexdistance{(!21.north east)}{(!22.north west)}{+(0,2cm)}{above}
%\measurexdistance{(!1.north east)}{(!221.north west)}{+(0,-2.4cm)}{below}
Auch die Äste können unterschiedlich gespreizt werden (\texttt{fixed edge angles}),
wie es auch möglich ist, die Kantenhöhe mit \verb|for tree={l=<Wert>cm}| zu verändern 
(Beispiel~\ref{forestexample:13})
\footnote{\cite[Beispiel~59]{zivanovic:forest}}.

\begin{lfgwexample}{label={forestexample:13}}
\begin{forest}
calign=fixed edge angles,
calign primary angle=-30,calign secondary angle=60,
for tree={l=2cm}
[CP[C][TP]]
\draw[dotted] (!1) -| coordinate(p) () (!2) -| ();
\path ()--(p) node[pos=0.4,left,inner sep=1pt]{-30};
\path ()--(p) node[pos=0.1,right,inner sep=1pt]{60};
\end{forest}
\end{lfgwexample}

\subsubsection{Klammerschreibung}

Die in der Weiterentwicklung der GG im \enquote{Minimalistischen Programm}
zentrale Operation Merge (Verkettung) erzeugt mengentheoretische Objekte, die
\enquote{strenggenommen keine Strukturbäume mehr darstellen}\footnote{Günter Grewendorf: Minimalistische Syntax. A. Francke: Tübingen und Basel 2002, S.~126}. Mit Klammerschreibungen in geschweiften (s. \ref{forestexample:12}) oder eckigen Klammern (siehe Beispiel \ref{forestexample:1} oben) können diese mithilfe der Pakete \texttt{amsmath,amssymb} und diversen Umgebungen einfach in \TeX\ gesetzt werden. Sie müssen
in eine mathematische Umgebung eingefügt werden, dies sind u.\,a.:

\begin{lstlisting}
$                 $ für kürze Formalisierungen
oder
\[                \] für längere Formalisierungen
\end{lstlisting}

\begin{lfgwprint}{label={forestexample:14}}
1. Schritt Verkettung von \emph{trinken} und \emph{Milch}:\\
$ K= \{trinkt,\{Milch,trikt\}\}\ $ \\
2. Schritt Verkettung von K und Pauline:\\
$ \{trinkt, \{Pauline, \{trinkt, \{trinkt, \{Milch, trinkt\}\}\}\}\} $
\end{lfgwprint}

\begin{lstlisting}
1. Schritt Verkettung von \emph{trinken} und \emph{Milch}:\\
$ K= \{trinkt,\{Milch,trikt\}\}\ $ \\
2. Schritt Verkettung von K und \emph{Pauline}:\\
$ \{trinkt, \{Pauline, \{trinkt, \{trinkt, \{Milch, trinkt\}\}\}\}\} $
\end{lstlisting}

Beim Setzen im mathematischen Modus werden alle Buchstaben kursiv gestellt und Leerzeichen
entfernt. Mit den Befehlen \verb|text{  }| und \verb|\enspace| kann man das abändern (s. \ref{forestexample:15}).

\begin{lfgwexample}{label={forestexample:15}}
Mary wants to know\\
i) $ [[_{wh} in which house]John lived t] $ \\
ii) $ [[_\text{wh} \enspace \text{in} \enspace \text{which} \enspace \text{house}]\enspace \text{John} \enspace \text{lived} \enspace \text{t}] $
\end{lfgwexample}

%% amsmath


%\end{document}




